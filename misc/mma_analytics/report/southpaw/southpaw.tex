\clearpage
\section{Orthodox vs. Southpaw Stance}
There have been 1641 fighters to fight in the UFC. We have stance information on 1503 of them. Here is the
breakdown:

\begin{table}[h]
\input{southpaw/stance_breakdown.tex}
\caption{Number and proportion of fighters with each stance type. About 10\% of the
world population is left handed according to Hardyck and Petrinovich, ``Left-handedness'', Psychol Bull. 84 (3): 385–404 (1977).}
\end{table}

Given that 10\% of the world population is left handed, it is surprising to see that 17.5\%
of UFC fighters are southpaws. Let's conduct a statistical test to see if the elevated
proportion could be explained by chance.
The standard error is $SE = \sqrt{\frac{p_0(1-p_0)}{n}}=\sqrt{\frac{0.1\times(1 - 0.1)}{1503}}\approx0.007$.
The corresponding test statistic is $Z = \frac{0.175 - 0.1}{SE}\approx9.7$. Using a two-sided test
we find $p \sim 10^{-22}$ which indicates that chance alone does not explain the elevated proportion of southpaws in the UFC.
For a soon to be explained reason, left-handed fighters are drawn to the fight game.

\begin{table}[h]
\input{southpaw/ortho_vs_south.tex}
\caption{List of fights with a victor between an orthodox and southpaw fighter. There are 1010 fights between 745 unique fighters
since January 1, 2005.}
\label{plain_south}
\end{table}

Is there an advantage to fighting in the orthodox versus southpaw stance?
As shown in Table~\ref{plain_south}, There have been 1010 fights between fighters with an orthodox and southpaw stance which resulted in a clear victor (i.e., we ignore draws and no contests). The orthodox fighters have won 449 (44.4\%) of these fights while the southpaws have won the balance of 561 (55.5\%). Is this result statistically significant? That is, could this outcome be explained by chance. Let's assume that the outcome of each fight is independent of stance. In this case, the likelihood of observing a split of 449 and 561 or more extreme values is given by a two-sided test using the binomial distribution:

\begin{equation}
p = 2\sum_{k=0}^{449} \frac{1010!}{k! (1010-k)!}(1/2)^{1010} \approx 0.001.
\end{equation}

\noindent
We see that the $p$-value is much less than 0.05 indicating that the outcome of a fight does depend on stance.

To isolate the effect of stance, we repeat the calculation for fights where the age difference between fighters
is 3 years or less and the reach difference is 3 inches or less. 
Now we find 137 orthodox wins to 191 of the southpaws (see Table~\ref{extra_south}). The $p$-value of the
appropriate two-sided test is $p \approx 0.003$ suggesting once again
that stance is a factor.

\begin{center}
\begin{table}[h]
\input{southpaw/ortho_vs_south_same_age_reach.tex}
\caption{List of fights with a victor between an orthodox and southpaw fighter where both fighters have
similar age and reach. There are 328 such fights since January 1, 2005.}
\label{extra_south}
\end{table}
\end{center}

\begin{figure}[h]
\begin{center}
\includegraphics[width=8cm]{southpaw/southpaw_win_ratio.pdf}
\caption{Win percentage by stance for fights where the reach difference
is 3 inches or less and the age difference is 3 years or less.}
\end{center}
\end{figure}

\begin{figure}[h]
\begin{center}
\includegraphics[width=11cm]{southpaw/stance_type_by_year.pdf}
\caption{Percentage of stance type versus year.}
\end{center}
\end{figure}
