\clearpage
\section{Effect of Age}
Table~\ref{biggest_age_diff} shows the fights
with the largest age differences in UFC history:

\begin{center}
\begin{table}[h!]
\input{age/biggest_age_diff.tex}
\caption{Biggest age differences in UFC history.}
\label{biggest_age_diff}
\end{table}
\end{center}

\clearpage

Note that 30 of these 40 fights or 75\% were won by the younger
fighter. This is a statistically
significant result ($p\sim 10^{-3}$). That is, if the win
probabilities of both fighters were the same, the outcome
of 30/40 would only occur with a likelihood of 1/1000.

What are the win percentages
of the younger and older fighter in all UFC fights?
There
are 3850 fights where both dates of birth are known
-- and not the same -- 
and the outcome is not a no contest.
We find
the younger fighter wins 54.8\% (2108) and the older
wins 44.6\% (1716). This result
is statistically significant ($p\sim 10^{-10}$).
The 26 draws in the data set explain why the two
win percentages do not sum to 100\%.

\begin{figure}[h]
\begin{center}
\includegraphics[width=8cm]{age/win_pct_younger_older.pdf}
\caption{Win percentage versus relative age. This calculation
was performed using all data from UFC 1 to UFC 208.}
\label{win_pct_younger_older}
\end{center}
\end{figure}

How do these win percentages change if we introduce
age brackets? That is, if only consider fights
between fighters who ages are between 18-24, 25-29,
30-34 and 35-39, does the younger fighter still win more?
Figure~\ref{win_pct_younger_older} suggests
that this is true for the middle age brackets. In the youngest
bracket age cancels out, and in the oldest, we find the
older fighter to have an advantage. However, none
of the differences in the four age brackets are statistically
significant. More data is needed to find out if these trends
are real.

\begin{figure}[h]
\begin{center}
\includegraphics[width=11cm]{age/age_brackets.pdf}
\caption{Win percentage of younger and older fighters
in different age brackets. The figure was constructed
using data beginning in 2005. Note that the results
are not statistically significant.}
\label{age_brackets}
\end{center}
\end{figure}

\clearpage

How do fighters in different age brackets achieve wins?
Table~\ref{finishes_by_age} shows the breakdown of
fight outcomes by age group. The older fighters tend to
win by KO/TKO while the younger fighters win more by
submission. 
If we ignore opponent disqualification then
one can apply a Chi-square test. We find
$\chi^2=22.3$ and $p \sim 10^{-3}$ with $\Phi_c=0.06$. This
indicates that differences in fight outcomes across
the groups are statistically significant.

\begin{center}
\begin{table}[h]
\input{age/finishes_by_age.tex}
\caption{How wins are achieved by age bracket since January 1, 2005.}
\label{finishes_by_age}
\end{table}
\end{center}

Is the mean age the same in every weight class for active UFC fighters?
Figure~\ref{boxplots_age} shows a box plot for fighter
ages for each weight class. It appears that heavyweight
has the largest median age. To answer the original
question we conducted a one-way ANOVA. Levene's test
gave $p=0.4$ which suggests that the variances of the data
sets are sufficiently similar. The ANOVA analysis yields
$F=5.8$ and $p \sim 10^{-8}$, which indicates that the means
are not equal.

\begin{figure}[h]
\begin{center}
\includegraphics[width=13cm]{age/anova_age_by_weightclass.pdf}
\caption{Box plots of fighter ages by weight class for active fighters. The horizontal dashed
line indicates the mean age of all active UFC fighters. The
numbers above each weight class label is the number of fighters
in that weight class. W--SW is women's strawweight, W--BW is women's
bantamweight, FYW is men's flyweight, BW is men's bantamweight, FTW
is men's featherweight, and so on.}
\label{boxplots_age}
\end{center}
\end{figure}

\clearpage

We see in Figure~\ref{win_pct_abs_age} that as age increases the win percentage decreases. The
correlation between age and win percentage is $-0.96$ by both
the Pearson and Spearman definitions ($p \ll 0.05$). A Chi-square
test showed $\chi^2=21.2$ and $p=0.39$ indicating that we cannot
conclude that win percentage is dependent on age.

\begin{figure}[h]
\begin{center}
\includegraphics[width=10cm]{age/win_percent_vs_age.pdf}
\caption{Win percentage versus age. The
solid line is a best fit. The dashed horizontal line indicates
a win percentage of 50\%. A 95\% confidence interval is shown for each point.}
\label{win_pct_abs_age}
\end{center}
\end{figure}

Win percentage of the younger fighter versus relative age
is shown in Figure~\ref{win_pct_rel_age}.
The Pearson correlation between age difference and win percentage of the
younger fighter is 0.84 with $p \ll 0.05$. A Chi-square
test showed $\chi^2=31.1$ with $p \approx 0.001$ and $\Phi_c=0.1$ indicating that win percentage
of the younger fight is dependent on age difference.

\begin{figure}[h]
\begin{center}
\includegraphics[width=10cm]{age/win_percent_of_younger.pdf}
\caption{Win percentage of the younger fighter versus the absolute
value of age difference. The
solid line is a best fit. The dashed horizontal line indicates
a win percentage of 50\%. A 95\% confidence interval is shown for each point.}
\label{win_pct_rel_age}
\end{center}
\end{figure}

\clearpage
\subsection*{Is ring rust real?}

Most fighters prefer to fight every 3 to 5 months. However, due to injury
and other factors
it is not uncommon
for the time between fights to exceed this range.

How often do individual fighters fight? Some fighters like Donald Cerrone
want to fight as soon as possible while others only step
into the octogon one or twice a year. The histogram of the number of months
between fights is shown in Figure~\ref{time_between_fights}. We see that the most
common time between fights is 4 months. 
Approximately 62\% of the total counts are between 3 and 6 months.
Note that four fighters have fought less than 16 days out from their
previous fight. And, of course, Royce Gracie fought four times in one night
to win UFC 2. However, here we only consider 2005 data and beyond.

\begin{figure}[h]
\begin{center}
\includegraphics[width=11cm]{age/time_between_fights.pdf}
\caption{Histogram of number of months between consecutive UFC fights
for individual fighters. Figure was constructed using all data since
January 1, 2005. A 95\% confidence interval is shown for each point.}
\label{time_between_fights}
\end{center}
\end{figure}

Figure~\ref{ring_rust} shows win percentage as a function
of time between fights averaged over all fighters.
For a quick turnaround of less than 3 months, the win percentage
is found to be $50\pm3.6\%$. This percentage is slightly
higher for 3 to 6 and 6 to 9 months.  Between 9 months
and 1 year is when a clear decline occurs. This continues
out to the 15-24 month interval where the win percentage
is only 39\%.
Note that in forming this figure we only consider UFC fights. If a fighter
leaves the UFC and fights in another promotion before returning, the
external fights are ignored. Chi-square analysis of the
data in Figure~\ref{ring_rust} yields
$\chi^2=21.1$, $p \sim 10^{-4}$ and $\Phi_c=0.06$, which
indicates that the differences in win percentage across the
time windows are statistically significant.

\begin{figure}[h]
\begin{center}
\includegraphics[width=11cm]{age/ring_rust.pdf}
\caption{Win percentage versus elapsed time since last fight. The figure
was constructed using all UFC fights since January 1, 2005. A
95\% confidence interval is shown for each data point.}
\label{ring_rust}
\end{center}
\end{figure}

\clearpage
\subsection*{Longevity and Youngest-Oldest Tables}

\begin{center}
\begin{table}[h]
\input{age/longest_ufc_tensure.tex}
\caption{Top 25 fighters who started with the UFC the longest ago and are still active today.}
\end{table}
\end{center}

\begin{center}
\begin{table}[h]
\input{age/youngest_oldest.tex}
\caption{The top 30 youngest and oldest UFC fighters of all-time.}
\end{table}
\end{center}
