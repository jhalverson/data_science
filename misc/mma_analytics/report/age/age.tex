\clearpage
\section*{Effect of Age}

Before looking at the effect of age on win percentage, we examine
some of the oldest and youngest fighters.

\begin{center}
\begin{table}[h]
\input{age/biggest_age_diff.tex}
\caption{Biggest age differences in UFC history.}
\label{biggest_age_diff}
\end{table}
\end{center}

Note that 36 of the 50 fights were won by the younger
fighter. This is a statistically
significant result ($p\sim 10^{-3}$).

\begin{center}
\begin{table}[h]
\input{longest_ufc_tensure.tex}
\caption{Top 25 fighters who started with the UFC the longest ago and are still active today.}
\end{table}
\end{center}

\begin{center}
\begin{table}[h]
\input{age/finishes_by_age.tex}
\caption{How wins are achieved by age bracket.}
\label{finishes_by_age}
\end{table}
\end{center}

Table~\ref{finishes_by_age}. If we ignore DQ then
one can apply a Chi-square test and find
$\chi^2=22.3$, $p=0.001$, $\Phi_c=0.06$.

We only consider fights between UFC 208 and January 1, 2005.

\begin{center}
\begin{table}[h]
\input{age/youngest_oldest.tex}
\caption{Top 25 fighters who started with the UFC the longest ago and are still active today.}
\end{table}
\end{center}

\clearpage

\begin{figure}[h]
\begin{center}
\includegraphics[width=8cm]{age/win_pct_younger_older.pdf}
\caption{Win percentage versus relative age.}
\label{win_pct_younger_older}
\end{center}
\end{figure}

\begin{figure}[h]
\begin{center}
\includegraphics[width=11cm]{age/age_brackets.pdf}
\caption{Win percentage of younger and older fighters
in different age brackets. Win percentage for same
age may be less than 50\% due to draws.}
\label{age_brackets}
\end{center}
\end{figure}

\begin{figure}[h]
\begin{center}
\includegraphics[width=13cm]{age/anova_age_by_weightclass.pdf}
\caption{Box plots of fighter ages by weight class. The horizontal dashed
line indicates the mean age of all active UFC fighters. The
numbers above each weight class label is the number of fighters
in that weight class. W--SW is women's strawweight, W--BW is women's
bantamweight, FYB is men's flyweight, BW is men's bantamweight, FTW
is men's featherweight, and so on.}
\end{center}
\end{figure}

ANOVA gives $F=5.8$ and $p \ll 0.05$ indicating that the mean age depends on weight class.

Number of wins by a fighter of age x divided by number of
fights that at least one combatant was age x. If both
combatants are age x then total possibilities is 2.

\clearpage

\begin{figure}[h]
\begin{center}
\includegraphics[width=10cm]{age/win_percent_vs_age.pdf}
\caption{We only show points for ages with more than 20 results. The
solid line is a best fit. The dashed horizontal line indicates
a win percentage of 50\%.}
\end{center}
\end{figure}

We see that as age increases the win percentage decreases. The
correlation between age and win percentage is $-0.96$ by both
the Pearson and Spearman definitions ($p \ll 0.05$). A Chi-square
test showed $\chi^2=21.2$ and $p=0.39$ indicating that we cannot
conclude that win percentage is dependent on age.

\begin{figure}[h]
\begin{center}
\includegraphics[width=10cm]{age/win_percent_of_younger.pdf}
\caption{We only show points for ages with more than 20 results. The
solid line is a best fit. The dashed horizontal line indicates
a win percentage of 50\%.}
\end{center}
\end{figure}

The Pearson correlation between age difference and win percentage of the
younger fighter is 0.84 with $p \ll 0.05$. A Chi-square
test showed $\chi^2=31.1$ with $p \approx 0.001$ and $\Phi_c=0.1$ indicating that win percentage
of the younger fight is not independent of age difference.

\subsection*{Is ring rust real?}

Most fighter prefer to fight every 3 to 5 months. However, due to injury,
contract negoiations and waiting for title fights, it is not uncommon
for the elapsed time between fights to exceed this range.

Figure~\ref{ring_rust} shows win percentage as a function
of time between fights averaged over all fighters and their fights.
The win percentage hovers around 50\% until around the one year
point where a decrease is seen. While the error bars are large, it
is clear that winning is a function of time from last fight.
Note that in forming this figure we only consider UFC fight. If a fighter
leaves the UFC and fights in another promotion before returning, the
external fights are ignored.

\begin{figure}[h]
\begin{center}
\includegraphics[width=11cm]{ring_rust.pdf}
\caption{Win percentage versus elapsed time since last fight. The figure
was constructed using all UFC fights since January 1, 2005. A
95\% confidence interval is shown for each data point.}
\label{ring_rust}
\end{center}
\end{figure}
