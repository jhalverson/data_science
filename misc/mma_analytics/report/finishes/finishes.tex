\clearpage
\section{Fight Outcomes}

Of the 4068 fights between UFC 1 and UFC 208,
a winner and loser were found in 3999,
42 were declared a no contest and 27 resulted in a draw.
Twelve of the 3999 wins were by opponent disqualification.

Figure~\ref{fight_outcome_by_year} shows the breakdown of fight
outcomes by year:

\begin{figure}[h]
\begin{center}
\includegraphics[width=11cm]{finishes/fight_outcome_breakdown.pdf}
\caption{Percentage of fight outcome by year for all UFC fights.}
\label{fight_outcome_by_year}
\end{center}
\end{figure}

The figure indicates that there have been fluctuations in how
fights end over time. Since about 2010, fight outcomes have
stabilized with KO/TKO at 30\%, submissions at 20\% and decision
around 50\%.
Other includes outcomes that were later overturned to a no contest,
disqualifications, CNC and other.

The methods of winning are shown as a percentage for each
of the ten weight classes in Figure~\ref{finishes_weight_class_bars}.
When opponent disqualification is removed, a Chi-square test
applied to the data in Figure~\ref{finishes_weight_class_bars}
showed $\chi^2=208$, $p \sim 10^{-34}$ and $\Phi_c=0.17$. This makes it
clear that the differences in the proportions across classes are
statistically significant.
For instance, the heavyweights have a KO/TKO percentage of 60\%.
A high KO/TKO percentage is also seen from the light heavyweight
to welterweight division.
Submissions account for roughly 20\% of finishes for each weight class with
decisions making up the balance. The women's strawweight has the
most fights go to decision at 65\%.

\begin{figure}[h]
\begin{center}
\includegraphics[width=14cm,right]{finishes/finishes_weight_class_bars.pdf}
\caption{Method of winning versus weight class since January 1, 2005.
Opponent DQ is very small and only visible in the MW and LH divisions.
W-–SW is women's strawweight, W-–BW is women's bantamweight, FYW is
men's flyweight, BW is men's bantamweight, FTW is men's
featherweight, and so on.}
\label{finishes_weight_class_bars}
\end{center}
\end{figure}

The percentage of total KO/TKO's and submissions versus
round is shown for each weight class in
Figure~\ref{submissions_tko_percentage}. We see most
finishes come in the first round and begin to diminish
for each round after. The data for the fourth and fifth
rounds are not weighted since we combined
three and five round fights.

\begin{figure}[h]
\begin{center}
\includegraphics[width=13cm]{finishes/submissions_tko_percentage.pdf}
\caption{(top) Percentage of total KO/TKO's for each
weight class by round and (bottom) the same except for submissions.}
\label{submissions_tko_percentage}
\end{center}
\end{figure}

%\begin{figure}[h]
%\begin{center}
%\includegraphics[width=12cm]{finishes/tko_sub_by_round.pdf}
%\caption{Breakdown of finishes by round for non-decision fights.}
%\label{height_reach_all_fighters}
%\end{center}
%\end{figure}
%A Chi-square test yields $\chi^2=8.5$ with $p=0.076$. This indicates
%that there is not enough evidence to argue that the finish technique
%depends on round.

\clearpage
\subsection*{Avoiding Two Consecutive Losses}

New UFC fighters are likely to get cut from the organization if they lose two
fights in a row. Similarly, fighters working towards a title shot are
face a significant setback if they drop two consecutive fights. Because of this
one might think that fighters fight harder when coming off a loss in
comparison to a win in their previous bout.

To test this hypothesis we looked at all fights since January 1, 2005 (ignoring no contest fights) and found
the following:

\begin{center}
\begin{table}[h]
\begin{tabular}{r|cc}
  \toprule
  & Lost previous & Won previous \\
  \hline
  Won subsequent & 1249 & 1699 \\
  Lost subsequent & 1226 & 1545 \\
  Draw subsequent & 12 & 20 \\
  \hline
  Total & 2487 & 3264 \\
  \bottomrule
\end{tabular}
\caption{Number of wins, losses and draws in the subsequent fight given a win or loss in a fighter's previous fight.}
\label{table_avoiding_two_losses}
\end{table}
\end{center}

We see in the left column of Table~\ref{table_avoiding_two_losses} that
when the previous fight was a loss, the subsequent fight resulted
in $n_w=1249$ wins, $n_l=1226$ losses and $n_d=20$ draws. There are more wins than losses but
is this result statistically significant or could it be explained
by chance? The probabily of a given outcome ($n_w, n_l, n_d$) is described by
the multinomial distribution:

\begin{equation}
f(n_i, p_i) = \frac{N!}{n_w! n_l! n_d!}p_w^{n_w}p_l^{n_l}p_d^{n_d},
\end{equation}

\noindent
where $\sum n_i = N$, $\sum p_i=1$ and $p_i$ is the probability
of event $i$. The $p$-value for a given
outcome $n_w, n_l, n_d=n_{i,0}$ is

\begin{equation}
p = \sum_{\substack{{\sum n_i = N} \\ {f \le f_0}}}f(n_i, p_i),
\label{cdf_multinomial}
\end{equation}

\noindent
where $f_0=f(n_{i,0})$.
There were 20 draws in the 3851 fights in our data set so we
choose $p_d=20/3851$. This determines the win and loss probabilities as
$p_w=p_l=1/2 - p_d/2$.
The probability of observing the outcome in the left
column of Table~\ref{table_avoiding_two_losses}, $f_0$, or a more extreme outcome
is given by Eq.~\ref{cdf_multinomial}. For the case where a fighter
is coming off a loss, we find $p=0.84$. A one-way Chi-square test gives $\chi^2=0.47$ with $p=0.79$.
Thus, we cannot conclude that a fighter is more likely to win their next fight
when coming off a loss.

In the case
of coming off a win (right column), we find $p=0.022$ which
indicates the differene in wins and losses is statistically significant. This is supported by a Chi-square test which
gives $\chi^2=7.48$ and $p=0.024$. Hence, 
UFC fighters are slightly more likely to win their next
fight (with win percentage of 52.1\%) when coming off a win.
We note that we could not find any software that provided the cumulative
distribution function for the multinomial distribution. R only provides
the probability mass function.

\subsection*{Outcome of UFC Debut}

Most fighters admit to being nervous before their first UFC fight.
Since January 1, 2005, there have been 1379 fighters to make their UFC debut. We found the
win, loss and draw percentages to be 43.3\%, 56.1\% and 0.6\%, respectively.
This result is statistically significant with Eq.~\ref{cdf_multinomial}
giving $p \sim 10^{-5}$. Thus, most
fighters lose their UFC debut. This is an overall analysis and it does not
take into account age or previous experience.


\subsection*{Most Finishes}
We finish this section with some interesting tables:

\begin{center}
\begin{table}[h]
\input{finishes/most_submissions.tex}
\caption{All-time list of UFC fighters with 5 or more submission finishes
as of February 11, 2017.}
\end{table}
\end{center}

\begin{center}
\begin{table}[h]
\input{finishes/most_tkos.tex}
\caption{All-time list of UFC fighters with 8 or more KO/TKO finishes
as of February 11, 2017.}
\end{table}
\end{center}

\begin{center}
\begin{table}[h]
\input{finishes/late_finishes.tex}
\caption{The only 5th round finishes in UFC history as of February 11, 2017.}
\label{latest_finishes}
\end{table}
\end{center}
