\clearpage
\section{Reach and Height}

We have reach and height data for 1210 of the 1641 fighters who have ever fought in the
UFC. These data are plotted in Figure~\ref{height_reach_all_fighters}:

\begin{figure}[h]
\begin{center}
\includegraphics[width=10cm]{reach_height/height_reach_all_fighters.pdf}
\caption{Height versus reach data for roughly 75\% of UFC fighters
of all-time.
Most of the data falls above the 45$^{\circ}$ line indicating that
reach tends to exceed height. The Pearson correlation between height
and reach is $r=0.89$ with $p\ll 0.05$. The data points have been randomly displaced
by a small amount to afford visualization.}
\label{height_reach_all_fighters}
\end{center}
\end{figure}

The tallest UFC fighter of all-time is Stefan Struve at 7 feet or 84 inches. The fighter
with the greatest reach is Semmy Schilt at 7 feet, 4 inches. It is interesting
to note that while Jon Jones is 8 inches shorter than
Stefan Struve, the two have the same reach of 84 inches. Figure~\ref{jones_reach} shows
how Jones stacks up against other light heavyweights who have competed in the UFC
over the past 24 years:

\begin{figure}[h]
\begin{center}
\includegraphics[width=14cm]{reach_height/jones_reach.pdf}
\caption{(left) Height and reach of 189 light heavyweights throughout
UFC history.
(right) Histogram of reach data from the panel on the left.}
\label{jones_reach}
\end{center}
\end{figure}

Table~\ref{biggest_reach_diff} shows all UFC fights where the reach difference
was 9 inches or greater. The fighter with the
reach advantage won 31 of the 45 fights or 69\%. Using the
binomial distribution it can be shown that this result is
statistically significant ($p\approx0.016$). That is, if one
assumes that reach difference is not a factor, it is unlikely to observe
such a split (31/45) by chance.

\begin{center}
\begin{table}[h]
\input{reach_height/biggest_reach_diff.tex}
\caption{All UFC fights with a reach difference of 9 inches or more.}
\label{biggest_reach_diff}
\end{table}
\end{center}

\clearpage

\begin{figure}[h]
\begin{center}
\includegraphics[width=8cm]{reach_height/longer_reach_win_ratio.pdf}
\caption{Win percentage in fights with a reach difference using
all available data.}
\label{winratio_reach_diff_simple}
\end{center}
\end{figure}

When there is a reach difference, does the fighter
with greater reach tend to win?
There are 2952 fights that meet this condition and
1546 were won by the fighter with longer reach while
1406 by the fighter with shorter reach. This difference
is statistically significant ($p=0.01$).
Figure~\ref{winratio_reach_diff_simple}
shows the win percentages for the two fighters.

Figure~\ref{winratio_reach_diff}
shows the win percentage of the fighter with
longer reach as a function of reach difference.
As reach difference increases, the win likelihood of the fighter with greater reach
increases.
A Chi-square test yields $\chi^2=20.2$, $p=0.017$ and $\Phi_c=0.08$.
This indicates that there is an association between the proportions of wins and losses  
and reach difference.

What happens when both fighters have the same reach but there is
a height difference? There are 327 fights of all-time that meet this condition
and the taller fighter won 160 while the shorter won 167. However, the
difference is not statistically significant ($p=0.74$). What if we ignore reach
and simply ask how often does the taller fighter win? We find of the 3282 fights
where the two fighters have different heights, the taller fighter wins 1683 to
1599 of the shorter. This difference is not statistically significant ($p=0.15$).

\begin{figure}[h]
\begin{center}
\includegraphics[width=11cm]{reach_height/winratio_reach_diff.pdf}
\caption{Win percentage of the fighter with longer reach versus reach difference. A 95\%
confidence interval is shown for each point. These intervals were computed
as $\pm t_{n-1}^{*}\sqrt{\frac{\hat{p}(1-\hat{p})}{n}}$, where $t_{n-1}^{*}$ is always somewhat greater than
1.96.}
\label{winratio_reach_diff}
\end{center}
\end{figure}

\begin{center}
\begin{table}[h]
\input{reach_height/reach2height_large.tex}
\caption{All-time largest and smallest reach-to-height ratios for UFC fighters.}
\label{reach2height_large}
\end{table}
\end{center}

How do fight outcomes vary with reach-to-height ratio? Let's first look
at the largest and smallest ratios in UFC history (see
Table~\ref{reach2height_large}).

\clearpage

Win percentage is plotted against reach-to-height ratio in
Figure~\ref{reach_vs_win_percent} for UFC
fighters with at least five fights since January 1, 2005.
We find the Pearson correlation between these quantities
to be $r=0.07$ with $p=0.09$. This suggests
a weak correlation that is not statistically significant.

\begin{figure}[h]
\begin{center}
\includegraphics[width=10cm]{reach_height/reach_vs_win_percent.pdf}
\caption{Win percentage versus reach-to-height ratio for UFC
fighters with at least five fights since January 1, 2005. The solid line
is a best fit to the data points.}
\label{reach_vs_win_percent}
\end{center}
\end{figure}

Because heigth and leg reach are collinear with reach, we will not
consider a detailed analysis of these quantities.

\clearpage
\subsection*{How has average reach, height and age for each weight class varied over time?}

Here we compute the average reach, height and age of UFC fighters
by weight class
for each year going all the way back to 1997. This section
contains some of the most striking results of this document.

Examination of Figs.~\ref{reach_vs_time_light} through
\ref{reach_vs_time_heavy} reveal that the average reach,
height and age of fighters within each weight class
have increased over the years. In some cases the changes
are dramatic.
For instance, the average age of a middleweight was
26 in 1998 but in 2016 it was 32! Also, middleweight reach
has increased by almost 2 inches. The same is true of
the welterweights.
The height of the average welterweight was around 70 inches in 2005.
Today it is about 71.5 inches with an average reach increase of
about 2 inches. It is only the heavyweight division that where
reach and height have not changed only the last decade. However,
the average age has increased.

The primary explanation for these trends must be advances made in
weight cutting. Many fighters hire a dietician to oversee their
diet and weight cut.

One interesting note is that there were no lightweight fights in the UFC
in 2005. The featherweights were incorporated from WEC in 2010.

%%%%%%%%%%%%%%%%
\begin{figure}[h]
\begin{center}
\begin{subfigure}{8cm}
\includegraphics[width=7.5cm]{age/featherweight_age_height_reach.pdf}
\end{subfigure}
\begin{subfigure}{8cm}
\includegraphics[width=7.5cm]{age/lightweight_age_height_reach.pdf}
\end{subfigure}
\caption{Average reach, height, age and total number
of fights
for the featherweight and lightweight divisions.
Reach and height are in units of inches.}
\label{reach_vs_time_light}
\end{center}
\end{figure}
%%%%%%%%%%%%%%%%

\clearpage
%%%%%%%%%%%%%%%%
\begin{figure*}
    \centering
    \begin{subfigure}[b]{0.45\textwidth}
	\centering
	\includegraphics[width=\textwidth,right]{age/welterweight_age_height_reach.pdf}
	\label{fig:mean and std of net14}
    \end{subfigure}
    \hfill
    \begin{subfigure}[b]{0.45\textwidth}  
	\centering 
	\includegraphics[width=\textwidth]{age/middleweight_age_height_reach.pdf}
	\label{fig:mean and std of net24}
    \end{subfigure}
    \vskip\baselineskip
    \begin{subfigure}[b]{0.45\textwidth}   
	\centering 
	\includegraphics[width=\textwidth,left]{age/light_heavyweight_age_height_reach.pdf}
	\label{fig:mean and std of net34}
    \end{subfigure}
    \hfill
    \begin{subfigure}[b]{0.45\textwidth}   
	\centering 
	\includegraphics[width=\textwidth,right]{age/heavyweight_age_height_reach.pdf}
	\label{fig:mean and std of net44}
    \end{subfigure}
    \caption{Average reach, height, age and total number of fights
for the welterweight, middleweight, light heavyweight and heavyweight divisions.
Reach and height are in units of inches.}
    \label{reach_vs_time_heavy}
\end{figure*}
%%%%%%%%%%%%%%%%
