\clearpage
\section{Conclusions}

The UFC has grown steadily since 2005. Today the company holds
more than 40 events per year in countries around the world. Their
women's divisions which are relatively new are as well established
as the men's divisions.

We have shown that the outcome of a UFC fight is strongly effected
by the relative age, reach,
stance, education level and time since last bout of the two fighters.
Southpaw fighters who are younger and have a reach advantage over their
orthodox opponents win 62 percent of the time.

Models were created to predict fight outcomes. Simple rule-based models
revealed the importance of
age, reach and stance in determining fight outcomes.
A variety of machine learning models were considered.
Using only the age and reach of the two fighters, one can predict
fight outcomes with
56.8\% accuracy using a logistic regression model. When many additional
features such as win ratio,
significant strikes landed per minute and takedown defense were
added, our accuracy only improved to about 60\%.
We argued that the explanation
for the limited improvement is that the values
of the additional features are not precisely known for fighters
with few UFC fights.
In 78\% of all UFC bouts at least one of the two fighters
had only four previous UFC fights or less. For these 
fighters, quantities such
as significant strikes landed per minute, $s_l$, are not well-established.
We demonstrated that on average about seven fights
are needed for $s_l$ to approach its career-average value. When
only fights where both combatants have 8 fights or more were
considered, our accuracy improved to 61.1\%, which is somewhat
close to the scores achieved when using career-average values
for eight key metrics. One could use data from other
promotions to more accurately determine the
key metrics for fighters with few UFC fights but
this is beyond the scope of this work.
