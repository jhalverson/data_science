\clearpage
\section{Education}

The UFC website provides data on the education of each fighter.
There are 186 UFC fighters with college or degree. Only 55
of the 533 active fighters have college experience.

The complete list of college data is found in Table~\ref{ed_table}
in Appendix~\ref{appendix_education}.

If a fighter in the UFC database is listed as having a degree
or college then we will consider them a college graduate. In
this section we look at all UFC fights between a college graduate
and a fighter who did not graduate college. Is there an advantage
to higher education?

Let's first look at the percentage of fights with at least
one college graduate as shown in Fig.~\ref{fights_with_college_grad}:

\begin{figure}[h]
\begin{center}
\includegraphics[width=11cm]{education/fights_with_college_grad.pdf}
\caption{Percentage of UFC fights where at least one of the fighters
is a college graduate as a function of year.}
\label{fights_with_college_grad}
\end{center}
\end{figure}

Fig.~\ref{fights_with_college_grad} shows a steady increase in
college graduates until around 2011 when it begins to decline.
It maximum is 45\% in 2011. For our analysis, we will consider
all fights in 2003 and beyond.

Note that we excluded fights involving the thirteen fighters
not found in the UFC database. We also ignored the five fights that resulted
in a draw.
Using this starting date, there are 1059 fights between a college
graduate and a fighter who did not graduate college. We find
597 of these fights or 56.4\% were won by the college graduate.
If education was not a factor then the likelihood of observing
this outcome or more extreme outcomes is $p\sim10^{-5}$. This
suggests that education does play a role in determining the outcome
of a fight.

\begin{figure}[h]
\begin{center}
\includegraphics[width=8cm]{education/win_pct_college_no_college.pdf}
\caption{Win percentage when one fighter is a college graduate and their
opponent is not.}
\label{win_pct_college_no_college}
\end{center}
\end{figure}
