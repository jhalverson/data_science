\documentclass[12pt]{article}

\renewcommand{\thesection}{\Roman{section}.} 
\renewcommand{\thesubsection}{\thesection.\Roman{subsection}}

\usepackage{enumitem}
\usepackage{subcaption}
\usepackage{etoolbox}

\BeforeBeginEnvironment{tabular}{\begin{center}\footnotesize}
\AfterEndEnvironment{tabular}{\end{center}}

\title{\Large \textbf{Statistical Analysis and Predictive Modeling of Mixed Martial Arts Fighting: From UFC 1 to UFC 208}}
\date{\Large \today}
\author{\Large Jonathan D. Halverson\\ \tiny halverson.jonathan at gmail}

\usepackage[export]{adjustbox}
\usepackage{amsmath}
\usepackage[in]{fullpage}
\usepackage{booktabs}
\usepackage{longtable}
\usepackage{graphicx}
\usepackage[font=footnotesize,margin=2cm]{caption}

\begin{document}
\maketitle
\noindent
\textbf{Summary} -- Here we analyze UFC data from 1993 through February 2017. The most
striking results relate to the increased age, reach and height of
most weight classes over the last ten years. We find southpaws have a significant advantage against
orthodox fighters, ring rust becomes real around 9-12 months, and fighters tend to lose
more than they win after around age 30. In the last section we show that
methods from artificial intelligence can be used to predict fight outcomes with 66\% accuracy.

\input{intro/intro.tex}
\clearpage
\section{Effect of Age}
Table~\ref{biggest_age_diff} shows the fights
with the largest age differences in UFC history:

\begin{center}
\begin{table}[h!]
\input{age/biggest_age_diff.tex}
\caption{Biggest age differences in UFC history.}
\label{biggest_age_diff}
\end{table}
\end{center}

\clearpage

Note that 30 of these 40 fights or 75\% were won by the younger
fighter. This is a statistically
significant result ($p\sim 10^{-3}$). That is, if the win
probabilities of both fighters were the same, the outcome
of 30/40 would only occur with a likelihood of 1/1000.

What are the win percentages
of the younger and older fighter in all UFC fights?
There
are 3850 fights where both dates of birth are known
-- and not the same -- 
and the outcome is not a no contest.
We find
the younger fighter wins 54.8\% (2108) and the older
wins 44.6\% (1716). This result
is statistically significant ($p\sim 10^{-10}$).
The 26 draws in the data set explain why the two
win percentages do not sum to 100\%.

\begin{figure}[h]
\begin{center}
\includegraphics[width=8cm]{age/win_pct_younger_older.pdf}
\caption{Win percentage versus relative age. This calculation
was performed using all data from UFC 1 to UFC 208.}
\label{win_pct_younger_older}
\end{center}
\end{figure}

How do these win percentages change if we introduce
age brackets? That is, if only consider fights
between fighters who ages are between 18-24, 25-29,
30-34 and 35-39, does the younger fighter still win more?
Figure~\ref{win_pct_younger_older} suggests
that this is true for the middle age brackets. In the youngest
bracket age cancels out, and in the oldest, we find the
older fighter to have an advantage. However, none
of the differences in the four age brackets are statistically
significant. More data is needed to find out if these trends
are real.

\begin{figure}[h]
\begin{center}
\includegraphics[width=11cm]{age/age_brackets.pdf}
\caption{Win percentage of younger and older fighters
in different age brackets. The figure was constructed
using data beginning in 2005. Note that the results
are not statistically significant.}
\label{age_brackets}
\end{center}
\end{figure}

\clearpage

How do fighters in different age brackets achieve wins?
Table~\ref{finishes_by_age} shows the breakdown of
fight outcomes by age group. The older fighters tend to
win by KO/TKO while the younger fighters win more by
submission. 
If we ignore opponent disqualification then
one can apply a Chi-square test. We find
$\chi^2=22.3$ and $p \sim 10^{-3}$ with $\Phi_c=0.06$. This
indicates that differences in fight outcomes across
the groups are statistically significant.

\begin{center}
\begin{table}[h]
\input{age/finishes_by_age.tex}
\caption{How wins are achieved by age bracket since January 1, 2005.}
\label{finishes_by_age}
\end{table}
\end{center}

Is the mean age the same in every weight class for active UFC fighters?
Figure~\ref{boxplots_age} shows a box plot for fighter
ages for each weight class. It appears that heavyweight
has the largest median age. To answer the original
question we conducted a one-way ANOVA. Levene's test
gave $p=0.4$ which suggests that the variances of the data
sets are sufficiently similar. The ANOVA analysis yields
$F=5.8$ and $p \sim 10^{-8}$, which indicates that the means
are not equal.

\begin{figure}[h]
\begin{center}
\includegraphics[width=13cm]{age/anova_age_by_weightclass.pdf}
\caption{Box plots of fighter ages by weight class for active fighters. The horizontal dashed
line indicates the mean age of all active UFC fighters. The
numbers above each weight class label is the number of fighters
in that weight class. W--SW is women's strawweight, W--BW is women's
bantamweight, FYW is men's flyweight, BW is men's bantamweight, FTW
is men's featherweight, and so on.}
\label{boxplots_age}
\end{center}
\end{figure}

\clearpage

We see in Figure~\ref{win_pct_abs_age} that as age increases the win percentage decreases. The
correlation between age and win percentage is $-0.96$ by both
the Pearson and Spearman definitions ($p \ll 0.05$). A Chi-square
test showed $\chi^2=21.2$ and $p=0.39$ indicating that we cannot
conclude that win percentage is dependent on age.

\begin{figure}[h]
\begin{center}
\includegraphics[width=10cm]{age/win_percent_vs_age.pdf}
\caption{Win percentage versus age. The
solid line is a best fit. The dashed horizontal line indicates
a win percentage of 50\%. A 95\% confidence interval is shown for each point.}
\label{win_pct_abs_age}
\end{center}
\end{figure}

Win percentage of the younger fighter versus relative age
is shown in Figure~\ref{win_pct_rel_age}.
The Pearson correlation between age difference and win percentage of the
younger fighter is 0.84 with $p \ll 0.05$. A Chi-square
test showed $\chi^2=31.1$ with $p \approx 0.001$ and $\Phi_c=0.1$ indicating that win percentage
of the younger fight is dependent on age difference.

\begin{figure}[h]
\begin{center}
\includegraphics[width=10cm]{age/win_percent_of_younger.pdf}
\caption{Win percentage of the younger fighter versus the absolute
value of age difference. The
solid line is a best fit. The dashed horizontal line indicates
a win percentage of 50\%. A 95\% confidence interval is shown for each point.}
\label{win_pct_rel_age}
\end{center}
\end{figure}

\clearpage
\subsection*{Is ring rust real?}

Most fighters prefer to fight every 3 to 5 months. However, due to injury
and other factors
it is not uncommon
for the time between fights to exceed this range.

How often do individual fighters fight? Some fighters like Donald Cerrone
want to fight as soon as possible while others only step
into the octogon one or twice a year. The histogram of the number of months
between fights is shown in Figure~\ref{time_between_fights}. We see that the most
common time between fights is 4 months. 
Approximately 62\% of the total counts are between 3 and 6 months.
Note that four fighters have fought less than 16 days out from their
previous fight. And, of course, Royce Gracie fought four times in one night
to win UFC 2. However, here we only consider 2005 data and beyond.

\begin{figure}[h]
\begin{center}
\includegraphics[width=11cm]{age/time_between_fights.pdf}
\caption{Histogram of number of months between consecutive UFC fights
for individual fighters. Figure was constructed using all data since
January 1, 2005. A 95\% confidence interval is shown for each point.}
\label{time_between_fights}
\end{center}
\end{figure}

Figure~\ref{ring_rust} shows win percentage as a function
of time between fights averaged over all fighters.
For a quick turnaround of less than 3 months, the win percentage
is found to be $50\pm3.6\%$. This percentage is slightly
higher for 3 to 6 and 6 to 9 months.  Between 9 months
and 1 year is when a clear decline occurs. This continues
out to the 15-24 month interval where the win percentage
is only 39\%.
Note that in forming this figure we only consider UFC fights. If a fighter
leaves the UFC and fights in another promotion before returning, the
external fights are ignored. Chi-square analysis of the
data in Figure~\ref{ring_rust} yields
$\chi^2=21.1$, $p \sim 10^{-4}$ and $\Phi_c=0.06$, which
indicates that the differences in win percentage across the
time windows are statistically significant.

\begin{figure}[h]
\begin{center}
\includegraphics[width=11cm]{age/ring_rust.pdf}
\caption{Win percentage versus elapsed time since last fight. The figure
was constructed using all UFC fights since January 1, 2005. A
95\% confidence interval is shown for each data point.}
\label{ring_rust}
\end{center}
\end{figure}

\clearpage
\subsection*{Longevity and Youngest-Oldest Tables}

\begin{center}
\begin{table}[h]
\input{age/longest_ufc_tensure.tex}
\caption{Top 25 fighters who started with the UFC the longest ago and are still active today.}
\end{table}
\end{center}

\begin{center}
\begin{table}[h]
\input{age/youngest_oldest.tex}
\caption{The top 30 youngest and oldest UFC fighters of all-time.}
\end{table}
\end{center}

\clearpage
\section{Reach and Height}
We have reach and height data for 1210 of the 1641 fighters who have ever fought in the
UFC. These data are plotted in Figure~\ref{height_reach_all_fighters}. Most of
the data falls above the 45$^{\circ}$ line indicating that
reach tends to exceed height. The Pearson correlation coefficient between height
and reach is $r=0.89$ with $p\ll 0.05$.

\begin{figure}[h]
\begin{center}
\includegraphics[width=10cm]{reach_height/height_reach_all_fighters.pdf}
\caption{Height versus reach data for roughly 75\% of UFC fighters
of all-time. The data points have been randomly displaced
by a small amount to afford visualization.}
\label{height_reach_all_fighters}
\end{center}
\end{figure}

The tallest UFC fighter of all-time is Stefan Struve at 7 feet or 84 inches. The fighter
with the greatest reach is Semmy Schilt at 7 feet, 4 inches. It is interesting
to note that while Jon Jones is 8 inches shorter than
Stefan Struve, the two have the same reach of 84 inches. Figure~\ref{jones_reach} shows
how Jones stacks up against other light heavyweights who have competed in the UFC
over the past 24 years.

\begin{figure}[h]
\begin{center}
\includegraphics[width=15cm]{reach_height/jones_reach.pdf}
\caption{(left) Height and reach of 189 light heavyweights throughout
UFC history.
(right) Histogram of reach data from the panel on the left.}
\label{jones_reach}
\end{center}
\end{figure}

Table~\ref{biggest_reach_diff} shows all UFC fights where the reach difference
was 9 inches or greater. The fighter with the
reach advantage won 31 of the 45 fights or 69\%. Using the
binomial distribution it can be shown that this result is
statistically significant ($p\approx0.016$). That is, if one
assumes that reach difference is not a factor, it is unlikely to observe
such a split (31/45) or one more extreme by chance.

\begin{center}
\begin{table}[h]
\input{reach_height/biggest_reach_diff.tex}
\caption{All UFC fights with a reach difference of 9 inches or more.}
\label{biggest_reach_diff}
\end{table}
\end{center}

\clearpage

\begin{figure}[h]
\begin{center}
\includegraphics[width=8cm]{reach_height/longer_reach_win_ratio.pdf}
\caption{Win percentage in fights with a reach difference using
all available data.}
\label{winratio_reach_diff_simple}
\end{center}
\end{figure}

When there is a reach difference, does the fighter
with greater reach tend to win?
There are 2952 fights that meet this condition and
1546 were won by the fighter with longer reach while
1406 by the fighter with shorter reach. This difference
is statistically significant ($p=0.01$).
Figure~\ref{winratio_reach_diff_simple}
shows the win percentages for the two fighters.

Figure~\ref{winratio_reach_diff}
shows the win percentage of the fighter with
longer reach as a function of reach difference.
As reach difference increases, the win likelihood of the fighter with greater reach
increases.
A Chi-square test yields $\chi^2=20.2$, $p=0.017$ and $\Phi_c=0.08$.
This indicates that there is an association between the proportions of wins and losses  
and reach difference.

What happens when both fighters have the same reach but there is
a height difference? There are 327 fights of all-time that meet this condition
and the taller fighter won 160 while the shorter won 167. However, the
difference is not statistically significant ($p=0.74$). What if we ignore reach
and simply ask how often does the taller fighter win? We find of the 3282 fights
where the two fighters have different heights, the taller fighter wins 1683 to
1599 of the shorter. This difference is not statistically significant ($p=0.15$).

\begin{figure}[h]
\begin{center}
\includegraphics[width=11cm]{reach_height/winratio_reach_diff.pdf}
\caption{Win percentage of the fighter with longer reach versus reach difference. A 95\%
confidence interval is shown for each point. These intervals were computed
as $\pm t_{n-1}^{*}\sqrt{\frac{\hat{p}(1-\hat{p})}{n}}$, where $t_{n-1}^{*}$ is always somewhat greater than
1.96.}
\label{winratio_reach_diff}
\end{center}
\end{figure}

\begin{center}
\begin{table}[h]
\input{reach_height/reach2height_large.tex}
\caption{All-time largest and smallest reach-to-height ratios for UFC fighters.}
\label{reach2height_large}
\end{table}
\end{center}

How do fight outcomes vary with reach-to-height ratio? Let's first look
at the largest and smallest ratios in UFC history (see
Table~\ref{reach2height_large}).

\clearpage

Win percentage is plotted against reach-to-height ratio in
Figure~\ref{reach_vs_win_percent} for UFC
fighters with at least five fights since January 1, 2005.
We find the Pearson correlation between these quantities
to be $r=0.07$ with $p=0.09$. This suggests
a weak correlation that is not statistically significant.
Because height and leg reach are collinear with reach, we will not
consider a detailed analysis of these quantities.

\begin{figure}[h]
\begin{center}
\includegraphics[width=10cm]{reach_height/reach_vs_win_percent.pdf}
\caption{Win percentage versus reach-to-height ratio for UFC
fighters with at least five fights since January 1, 2005. The solid line
is a best fit to the data points.}
\label{reach_vs_win_percent}
\end{center}
\end{figure}


\clearpage
\subsection*{How has average reach, height and age for each weight class varied over time?}

Here we compute the average reach, height and age of UFC fighters
by weight class
for each year going all the way back to 1997. This section
contains some of the most striking results of this study.

Examination of Figures~\ref{reach_vs_time_light} and
\ref{reach_vs_time_heavy} reveal that the average reach,
height and age of fighters within each weight class
have increased over the years. In some cases the changes
are dramatic.
For instance, the average age of a middleweight was
26 in 1998 but in 2016 it was 32. Also, middleweight reach
has increased by almost 2 inches. The same is true of
the welterweights.
The height of the average welterweight was around 70 inches in 2005.
Today it is about 71.5 inches with an average reach increase of
about 2 inches. It is only the heavyweight division where
reach and height have not changed over the last decade. However,
the average age has increased. What's most amazing about increases
in average age, reach and height for the lower weight classes is
that there is no sign of leveling off.

The primary explanation for these trends must be advances made in
weight cutting. Many fighters hire a dietician to oversee their
diet and weight cut.

One interesting note is that there were no lightweight fights in the UFC
in 2005. And the featherweights were incorporated from WEC in 2010.

%%%%%%%%%%%%%%%%
\begin{figure}[h]
\begin{center}
\begin{subfigure}{8cm}
\includegraphics[width=7.5cm]{age/featherweight_age_height_reach.pdf}
\end{subfigure}
\begin{subfigure}{8cm}
\includegraphics[width=7.5cm]{age/lightweight_age_height_reach.pdf}
\end{subfigure}
\caption{Average reach, height, age and total number
of fights
for the featherweight and lightweight divisions.
Reach and height are in units of inches.}
\label{reach_vs_time_light}
\end{center}
\end{figure}
%%%%%%%%%%%%%%%%

\clearpage
%%%%%%%%%%%%%%%%
\begin{figure*}
    \centering
    \begin{subfigure}[b]{0.45\textwidth}
	\centering
	\includegraphics[width=\textwidth,right]{age/welterweight_age_height_reach.pdf}
	\label{fig:mean and std of net14}
    \end{subfigure}
    \hfill
    \begin{subfigure}[b]{0.45\textwidth}  
	\centering 
	\includegraphics[width=\textwidth]{age/middleweight_age_height_reach.pdf}
	\label{fig:mean and std of net24}
    \end{subfigure}
    \vskip\baselineskip
    \begin{subfigure}[b]{0.45\textwidth}   
	\centering 
	\includegraphics[width=\textwidth,left]{age/light_heavyweight_age_height_reach.pdf}
	\label{fig:mean and std of net34}
    \end{subfigure}
    \hfill
    \begin{subfigure}[b]{0.45\textwidth}   
	\centering 
	\includegraphics[width=\textwidth,right]{age/heavyweight_age_height_reach.pdf}
	\label{fig:mean and std of net44}
    \end{subfigure}
    \caption{Average reach, height, age and total number of fights
for the welterweight, middleweight, light heavyweight and heavyweight divisions.
Reach and height are in units of inches.}
    \label{reach_vs_time_heavy}
\end{figure*}
%%%%%%%%%%%%%%%%

\clearpage
\section{Orthodox vs. Southpaw Stance}

Is there an advantage to fighting in the orthodox versus southpaw stance?
There have been 1641 fighters. We had stance info on 1503 of them. Here is the
breakdown:

\begin{table}[h]
\input{southpaw/stance_breakdown.tex}
\caption{Number and proportion of fighters with each stance type. About 10\% of the
world population is left handed. Hardyck C, Petrinovich LF (1977). "Left-handedness". Psychol Bull. 84 (3): 385–404. doi:10.1037/0033-2909.84.3.385. PMID 859955.}
\end{table}

Given that 10\% of the world population is left handed, it is surprising to see that 17.5\%
of UFC fighters are southpaws. Let's conduct a statistical test to see if the elevated
proportion could be explained by chance.
The standard error is $SE = \sqrt{\frac{p_0(1-p_0)}{n}}=\sqrt{\frac{0.1\times(1 - 0.1)}{1503}}\approx0.007$.
The corresponding test statistic is $Z = \frac{0.175 - 0.1}{SE}\approx9.7$. Using a two-sided test
we find $p \ll 0.05$ which indicates that chance alone does not explain the elevated proportion of southpaws in the UFC.
As we will see, left-handed fighters tend to do well so they are drawn to the fight game.

\begin{table}[h]
\input{southpaw/ortho_vs_south.tex}
\caption{List of fights with a victor between an orthodox and southpaw fighter. There are 1010 fights going back to 2005. There are 745 unique fighters on the list.}
\end{table}

There have been 1010 fights between fighters with an orthodox and southpaw stance which resulted in a clear victor (i.e., we ignore draws and no contests). The orthrodox fighters have won 449 (44.4\%) of these fights while the southpaws have won the balance of 561 (55.5\%). Is this result statistically significant? That is, could this outcome be explained by chance. Let's assume that the outcome of each fight is independent of stance. In this case, the likelihood of observing a split of 449 and 561 or more extreme values is given by a two-sided test using the binomial distribution:

\begin{equation}
p = 2\sum_{k=0}^{449} \frac{1010!}{k! (1010-k)!}(1/2)^{1010} \approx 0.001.
\end{equation}

We see that the p-value is much less than 0.05 indicating that the outcome of a fight does depend on stance.

To isolate the effect of stance, we repeat the calculation for fights where the age difference between fighters
is 3 years or less and the reach difference is 3 inches or less. 

\begin{center}
\begin{table}[h]
\input{southpaw/ortho_vs_south_same_age_reach.tex}
\caption{List of fights with a victor between an orthodox and southpaw fighter. There are 328 fights going back to 2005. There are 365 unique fighters on the list.}
\end{table}
\end{center}

Now we find 137 orthodox wins to 191 of the southpaws. The p-value of the
appropriate two-sided test is $p \approx 0.003$ suggesting once again
that stance is a factor.

\begin{figure}[h]
\begin{center}
\includegraphics[width=8cm]{southpaw/southpaw_win_ratio.pdf}
\caption{Win percentage by stance for fights where the reach difference
is less than 3 inches and the age difference is less than 3 years.}
\end{center}
\end{figure}

We find 21.9\% of ranked fighters (160) are lefty while 20.9\% are lefty among ranked fighters (533).

\begin{figure}[h]
\begin{center}
\includegraphics[width=11cm]{southpaw/stance_type_by_year.pdf}
\caption{Percentage of stance type versus year.}
\end{center}
\end{figure}

\clearpage
\section*{Finishes}

There have only been thirteen fifth round finishes in the history of the UFC.
These are listed in Table~\ref{latest_finishes}:

\begin{center}
\begin{table}[h]
\input{most_submissions.tex}
\caption{All-time list of fighters with 5 or more submission finishes.}
\end{table}
\end{center}

\begin{center}
\begin{table}[h]
\input{most_tkos.tex}
\caption{All-time list of fighters with 8 or more KO/TKO finishes.}
\end{table}
\end{center}

\begin{figure}[h]
\begin{center}
\includegraphics[width=12cm]{fight_outcome_breakdown.pdf}
\caption{Outcome of fights.}
\end{center}
\end{figure}

In Table~\ref{finishes_by_weight_class}, We removed DQ and Chi-square yields $\chi^2=177.2$, $p\ll0.05$,$\Phi_c=0.18$.

\begin{figure}[h]
\begin{center}
\includegraphics[width=14cm]{finishes/finishes_weight_class_bars.pdf}
\caption{Outcome of fights.}
\label{finishes_weight_class_bars}
\end{center}
\end{figure}

\begin{center}
\begin{table}[h]
\input{finishes/finishes_by_weight_class.tex}
\caption{How wins are achieved by weight class since 2005.}
\label{finishes_by_weight_class}
\end{table}
\end{center}

\begin{figure}[h]
\begin{center}
\includegraphics[width=13cm]{finishes/submissions_tko_percentage.pdf}
\caption{Outcome of fights.}
\label{submissions_tko_percentage}
\end{center}
\end{figure}

\begin{center}
\begin{table}[h]
\input{finishes/late_finishes.tex}
\caption{The only fifth round finishes in UFC history.}
\label{latest_finishes}
\end{table}
\end{center}

\begin{figure}[h]
\begin{center}
\includegraphics[width=12cm]{finishes/tko_sub_by_round.pdf}
\caption{Breakdown of finishes by round for non-decision fights.}
\label{height_reach_all_fighters}
\end{center}
\end{figure}

A Chi-square test yields $\chi^2=8.5$ with $p=0.076$. This indicates
that there is not enough evidence to argue that the finish technique
depends on round.

\clearpage
\section*{Education}

The UFC website provides data on the education of each fighter.
There are 189 UFC fighters with college or degree. Only 55
of the 533 active fighters have college experience.

The complete list of college data is found in Table~\ref{ed_table}
in Appendix~\ref{appendix_education}.

If a fighter in the UFC database is listed as having a degree
or college then we will consider them a college graduate. In
this section we look at all UFC fights between a college graduate
and a fighter who did not graduate college. Is there an advantage
to higher education?

Let's first look at the percentage of fights with at least
one college graduate as shown in Fig.~\ref{fights_with_college_grad}:

\begin{figure}[h]
\begin{center}
\includegraphics[width=11cm]{education/fights_with_college_grad.pdf}
\caption{Percentage of UFC fights where at least one of the fighters
is a college graduate as a function of year.}
\label{fights_with_college_grad}
\end{center}
\end{figure}

Fig.~\ref{fights_with_college_grad} shows a steady increase in
college graduates until around 2011 when it begins to decline.
It maximum is 45\% in 2011. For our analysis, we will consider
all fights in 2003 and beyond.

Note that we excluded fights involving the thirteen fighters
not found in the UFC database. We also ignored the five fights that resulted
in a draw.
Using this starting date, there are 1059 fights between a college
graduate and a fight who did not graduate college. We find
597 of these fights or 56.4\% were won by the college graduate.
If education was not a factor then the likelihood of observing
this outcome or more extreme outcomes is $p\sim10^{-5}$. This
suggests that education does play a role in determining the outcome
of a fight.

\begin{figure}[h]
\begin{center}
\includegraphics[width=9cm]{education/win_pct_college_no_college.pdf}
\caption{Win percentage when one fighter is a college graduate and their
opponent is not.}
\label{win_pct_college_no_college}
\end{center}
\end{figure}

\input{offense_defense/offense_defense.tex}
\clearpage
\section{Predicting the Winner}

In this section we create several models to try to predict
the winner. We start with simple models based on rules
and then move on to machine learning models such as
random forests, artificial neural networks and
adaptive boosting. It will be shown that prediction is quite
difficult with our best model having an accuracy of
only about 60\%.

\subsection*{Simple Rules}

If we exclude no contests and draws, since January 1, 2005,
there have been 3561 UFC fights with a winner and a loser.
Table~\ref{table_simple_rules} shows the accuracy score
for several simple rule-based models. For instance, if
our model is that the younger fighter always wins, we would
be correct 1964 times out of 3529 fights or 55.7\%.
There are 32 fights where either the dates of birth
are unknown or both fighters have the same date of birth so
these are excluded in the calculation.

\begin{center}
\begin{table}[h]
\begin{tabular}{r|ccccc}
  \toprule
  Model & Accuracy & Wins & Total & Excluded & $\log_{10}p$\\
  \hline
  Lesser age wins & 55.7\% & 1964 & 3529 & 32 & $-11$\\
  Lesser age by 4 years or more wins & 59.1\% & 893 & 1511 & 2050 & $-12$\\
  Greater reach wins & 52.4\% & 1499 & 2863 & 698 & $-2$\\
  Greater reach by 4 inches or more wins & 55.5\% & 523 & 942 & 2619 & $-4$\\
  Greater height wins & 51.2\% & 1486 & 2903 & 658 & $-0.7$\\
  Greater leg reach wins & 49.0\% & 487 & 993 & 2568 & $-0.2$\\
  Southpaw (vs. orthodox) wins & 55.5\% & 561 & 1010 & 2551 & $-4$\\
  Education (vs. no education) wins & 56.5\% & 594 & 1051 & 2510 & $-5$\\
  Greater number of UFC fights wins & 55.6\% & 1606 & 2886 & 675 & $-9$\\
  Fighter who fought more recently wins & 51.8\% & 1220 & 2353 & 1208 & $-1.1$\\
  Greater previous win ratio wins & 58.6\% & 434 & 740 & 2821 & $-6$\\
  Current or former champion wins & 61.8\% & 175 & 283 & 3278 & $-5$\\
  \hline
  Lesser age and greater reach wins & 57.2\% & 919 & 1607 & 1954 & $-9$\\
  Southpaw and lesser age wins & 61.7\% & 284 & 460 & 3101 & $-7$\\
  Southpaw, lesser age and greater reach wins & 62.0\% & 132 & 213 & 3348 & $-4$\\
  Greater previous win ratio and lesser age wins & 61.9\% & 252 & 407 & 3154 & $-6$\\
  \bottomrule
\end{tabular}
\caption{Accuracy scores for simple rule-based models. For models where win ratio was used, we only considered
fighters with 5 or more UFC fights. $p$-values are given in the rightmost column. Note that for $p=0.05$, $\log_{10}p = -1.3$. In all
cases the sum of 
Total and Excluded is 3561.}
\label{table_simple_rules}
\end{table}
\end{center}

In computing the number of previous UFC fights for each fighter, we start
counting with UFC 1. We ignored no contests but draws were included. A fighter's
win ratio is their number of UFC wins divided by their
total number of UFC fights (including draws).
Table~\ref{table_simple_rules} makes clear the advantage of being young, having
greater reach and being a southpaw.  Note that Conor McGregor has all three of
these traits.

\subsection*{Machine Learning Models}

One limitation of simple rule-based models is that they
cannot be applied to all fights. For instance, if the two fighters
have the same reach, a model that predicts the fighter with the longer
reach to win can say nothing.
In this section we go beyond simple rules to machine learning,
which is a branch of Artificial Intelligence.

In the first stage of supervised machine learning, a model
is trained by feeding it information about the two fighters
and which of the two won the bout. After the training phase,
the model is said to be smart and we show it information about
two fighters and ask it to tell us who won.

For each fighter we have the following features:

\begin{itemize}[noitemsep]
  \item Height, $h$
  \item Reach, $r$
  \item Leg reach, $l$
  \item Stance (orthodox, southpaw or switch), $osw$
  \item Date of birth or age, $a$
  \item College graduate, $e$
  \item Current or former champion/interim champion, $c$
  \item Number of UFC fights, $n_f$
  \item Number of UFC wins, $n_w$
  \item Win ratio of UFC fights, $w_r=n_w/n_f$
  \item Total time in the Octogon, $t_T$
  \item Elapsed time since last UFC fight, $t_e$
  \item Fraction of wins by KO/TKO, $W_{\textrm{TKO}}$
  \item Fraction of wins by submission, $W_{\textrm{SUB}}$
  \item Fraction of wins by decision, $W_{\textrm{DEC}}$
  \item Fraction of losses by KO/TKO, $L_{\textrm{TKO}}$
  \item Fraction of losses by submission, $L_{\textrm{SUB}}$
\end{itemize}

\noindent
We also use the following offensive and defensive statistics:

\begin{itemize}[noitemsep]
  \item Significant strikes landed per minute, $s_l$
  \item Significant striking accuracy, $a_s$
  \item Significant strikes absorbed per minute, $s_a$
  \item Significant strike defense, $a_f$
  \item Average takedowns landed per 15 minutes, $d_t$
  \item Takedown accuracy, $d_a$
  \item Takedown defense, $d_f$
  \item Average submissions attempted per 15 minutes, $s_b$
\end{itemize}

\begin{itemize}[noitemsep]
  \item Significant strikes attepted per minute
  \item Knockdowns per minute
  \item Total strikes landed per minute
  \item Total strikes attempted per minute
  \item Total strikes absorbed per minute
  \item Average takedowns attempted per 15 minutes
  \item Average passes per 15 minutes (e.g., half-mount to side-mount)
  \item Average reversals per 15 minutes (e.g., half-guard to full-guard)
\end{itemize}

\noindent
The derived features are

\begin{itemize}[noitemsep]
  \item Difference in age, reach and so on of the two fighters (e.g., $\Delta r=r_1-r_2$)
  \item Total damage absorbed = cumulative sum of significant strikes absorbed
  \item Striking ratio = significant strikes landed / significant strikes absorbed
  \item Natural logarithm of elapsed time since last UFC fight
  \item Fighter versatility index
\end{itemize}

\noindent
The fighter versatility index was introduced by Estelami as
\begin{equation}
FVI = \frac{\frac{1/3}{W_{\textrm{TKO}}^2+W_{\textrm{SUB}}^2+W_{\textrm{DEC}}^2}-\frac{1}{3}}{2/3}.
\end{equation}

The $FVI$ is zero when a fighter achieves all wins in one manner and
it is one when all three methods of victory are uniformly used.

\noindent
These features depend on both fighters:

\begin{itemize}[noitemsep]
  \item Expected strikes absorbed per minute of fighter 1 = $a_{f,1} (s_{l,2}/a_{s,2})$
  \item TKO susceptibility of fighter 1 = $(L_{\textrm{TKO},1} \times W_{\textrm{TKO},2})^{1/2}$
  \item Submission susceptibility of fighter 1 = $(L_{\textrm{SUB},1} \times W_{\textrm{SUB},2})^{1/2}$
  \item Same three above except for fighter 2
\end{itemize}

It is important to note that these quantites are computed
up to the time of the fight. We do not use any information
obtained during the fight or afterwards to make predictions.
Later, we will violate this rule to see if such an approach
leads to an improvement in model performance.

The most important part of any machine learning model is
to assemble a collection of discriminatory features. While
our list is comprehensive, we will find it to be insufficient.

One additional thought not taken into account is to weight
wins, losses, strikes landed and attempt by who the opponent
is. Such weights would need to be found in self-consistent
manner which suggest a cascading scheme.

Some of the featurs are correlated such as reach
and height. We will address this in the next section.


\subsection*{Prediction}

In cases where heigh and reach were known but leg reach
was missing, we imputed values by creating a linear model
based on reach and height. After removing fights where
dates of birth were not known, we were left with 3300 fights,
all of which took place after January 1, 2005.

Before creating our first model we examine a correlation
matrix of certain features. Table~\ref{corr_mat} shows that
most correlation coefficients are small. The largest ones
are for reach and height. This suggests that training a model
on both features will not be useful and may even be
detrimental. We see that there is a postive
correlation between win ratio and significant strikes landed
per minute while the correlation is negative between win ratio
and significant strikes absorbed per minute.

\begin{center}
\begin{table}[h]
\input{prediction/corr_table.tex}
\caption{Correlation matrix for basic features.}
\label{corr_mat}
\end{table}
\end{center}

During the training phase, we randomly select 70\% of the fights
and use this data to fit the models. This is done using
stratified K-fold cross validation with 10 folds. A grid search
procedure is used to find the optimal hyperparameters for each
model. The optimized model is then used to predict the outcome
of the remaining 30\% of the data. The accuracy is then determined
by comparing the predicted outcomes with the known outcomes.
Because results vary with train-test split, we shuffle the
data and repeat the procedure above ten times. The average
of these ten accuracy scores are then recorded.
While using the
test data is consider bad practice, this approach eliminates
concerns associated with using a single split which may give
favorable or unfavorable results depending on how the data
is partitioned.

Random forests (RF) is a tree-based ensemble method,
logistic regression (LR) is an example of the generalized
linear model with the logit link function, the multilayer perceptron (MLP)
is an artificial neural network and AdaBoost is
an ensemble method based on adaptive boosting.
For each model we optimize the hyperparameters
during K-fold cross-validation. For RF the
number of estimators is 100, bootstrapping is used
and the number of features to split on is taken
as the square root of the total number of features.
For LR we optimize the regularization coefficient
as well as the type of regularization ($L_1$ or $L_2$).
The sizes of the hidden layers were set to (5, 2) for
the MLP and $\alpha$, the $L_2$ regularization parameter, was optimized.
Lastly, for AdaBoost, the depth of the decision trees
(base classifiers), the learning rate and the number
of estimators were optimized.
For non-tree based methods, it is necessary to standardize
each feature. The standardizer was fit to the train data
and then used to transform both the train and test data.

Our first set of models is based on age, height,
reach, leg reach and stance. The accuracy scores
for different models and feature sets are show
in Table~\ref{anthropomorphic_features}.
We see that when age is the only feature, LR is gives
an accuracy of 55.6\%. This is very close to the simple
rule model as expected. The best we can do is just more than
56\%. Given the strong correlation between height, reach
and leg reach, it is not surprising that results do not
improve when these features are used together. We see that
age is a discriminatory feature.

\begin{center}
\begin{table}[h]
\begin{tabular}{r|cccc}
  \toprule
   & Random Forest & Logistic Reg. & Multilayer Perceptron \\
  Features & Accuracy & Accuracy& Accuracy \\
  \hline
  $a$ & 53.4\% & 55.6\% & 55.5\%\\
  $r$ & 52.2\% & 52.1\% & 51.9\%\\
  $a$, $r$ & 52.7\% & 56.8\% & 56.1\%\\
  $\Delta a$, $\Delta r$ & 52.9\% & 56.3\% & 56.2\%\\
  $h$, $r$ & 52.9\%  & 52.1\% & \\
  $\Delta h$, $\Delta r$ & 51.6\% & 51.6\%  &\\
  $h$, $\Delta h$, $r$, $\Delta r$ & 52.0\% & 51.9\%  &\\
  $h$, $r$, $l$ & 56.6\% & 51.9\% &  \\
  $\Delta h$, $\Delta r$, $\Delta l$ & 50.7\% & 52.2\%  &\\
  $\Delta h$, $\Delta r$, $\Delta l$, $\Delta a$ & 52.4\% & 55.1\% & 55.4\% \\
  $h$, $r$, $a$ & 53.4\% & 54.7\% &  \\
  $h$, $r$, $a$, $a^2$ & 53.4\% & 55.1\% &  \\
  $r/h$, $a^2$ & 54.3\% & 55.1\% & 55.2\%  \\
  $h$, $r$, $l$, $a$ & 56.1\% & 54.9\% &  \\
  $h$, $r$, $l$, $a$, $\Delta h$, $\Delta r$, $\Delta l$, $\Delta a$ & 55.8\% & 54.9\%  &\\
  $osw$ & 52.3\% & 52.3\% & 52.3\%\\
  $a$, $osw$ & 51.0\% & 55.0\% & \\
  $r$, $a$, $osw$ & 53.5\% & 55.7\% &  \\
  $r$, $l$, $a$, $osw$ & 56.1\% & 55.4\% & \\
  $h$, $r$, $l$, $a$, $osw$ & 56.7\% & 55.3\% &  \\
  $h$, $r$, $l$, $a$, $osw$, $\Delta h$, $\Delta r$, $\Delta l$, $\Delta a$, $\Delta osw$ & 56.0\% & 54.8\% &  \\
  \bottomrule
\end{tabular}
\caption{Accuracy scores for different machine learning classifiers which were trained
on age, stance and anthropometric measurements only.
$a$ is age, $h$ is height, $r$ is reach, $l$ is leg reach,
$osw$ stands for three indicator variables for orthodox, southpaw,
switch and $\Delta$ is the difference in one of the features between the two
fighters. For instance, in the third row the models were trained on
four features or the height and reach of both fighters (i.e., $h_1$, $r_1$, $h_2$, $r_2$)
whereas in the fourth row only two features were used (i.e., $h_1-h_2$ and $r_1-r_2$).}
\label{anthropomorphic_features}
\end{table}
\end{center}

Table~\ref{other_features} gives results for models with more
advanced choices of the features. We see that LR with win ratio
and age as the features (i.e., $w_{r,1}$, $w_{r,2}$, $a_1$, $a_2$) the
accuracy score is 57.5\%. As more featurs are introduced such as
total number of fights, stance, champion status and education, the
accuracy increases somewhat. We only achieve 60\% when the feature
extraction technique called linear discriminant analysis is applied
to the full feature matrix and then LR is used.

\begin{center}
\begin{table}[h]
\begin{tabular}{r|cccc}
\toprule
         & RF       & Logistic Reg. & MLP      & AdaBoost\\
Features & Accuracy & Accuracy      & Accuracy & Accuracy\\
\hline
$w_r$ & 54.0\% & 55.0\% & 55.9\% & \\
$\Delta w_r$ & 52.1\% & 55.5\% & 55.5\% & \\
$w_r$, $n_f$ & 52.7\% & 55.1\% & 53.7\% & \\
$w_r$, $n_f$, $n_w$ & 53.4\% & 55.8\% &  & \\
$w_r$, $a$ & 55.5\% & 57.5\% & 56.6\% & \\
$\Delta w_r$, $\Delta a$ & 53.3\% & 57.6\% & 57.3\% & \\
$n_f$, $n_w$, $a$, $r$ & 55.1\% & 57.0\% \\
$n_f$, $n_w$, $a$, $r$, $h$, $l$, $c$ & 57.8\% & 58.0\% & &\\
$n_f$, $n_w$, $a$, $r$, $h$, $l$, $osw$ & 57.2\% & 58.8\% & & 58.1\%\\
$n_f$, $n_w$, $a$, $r$, $h$, $l$, $osw$, $c$, $e$ & 58.0\% & 58.8\% & &\\
$n_f$, $n_w$, $a$, $r$, $h$, $l$, $osw$, $c$, $e$, $s_a$ & 58.0\% & 58.5\% & & \\
$n_f$, $n_w$, $a$, $r$, $h$, $l$, $osw$, $c$, $e$, $s_a$, $s_l$ & 57.9\% &59.1\% &58.9\% & \\
$n_f$, $n_w$, $a$, $r$, $h$, $l$, $osw$, $c$, $e$, $a_s$ & 58.6\% & 58.4\% \\
$n_f$, $n_w$, $a$, $r$, $h$, $l$, $osw$, $c$, $e$, $\Delta s_r$ & 57.8\% & 58.2\% & 58.4\% & 58.1\%\\
$n_f$, $n_w$, $a$, $r$, $osw$, $c$ & 56.6\% & 58.9\% & 58.5\% \\
$n_f$, $n_w$, $a$, $r$, $osw$, $c$, $h$, $l$ & 57.9\% & 59.0\% & &\\
$a$, $r$, $s_l$ & 54.9\% & 56.2\% & & \\
$a$, $r$, $d_t$ & 54.3\% & 56.1\% & & \\
$a$, $r$, $s_a$ & 55.7\% & 57.1\% & & \\
$s_l$, $s_a$ & 52.9\% & 56.8\% & 56.5\% & \\
all 121 features & 58.1\% & 59.3\% & 59.1\% & \\
all 121 features + LDA &  & 60.1\% & & \\
\bottomrule
\end{tabular}
\caption{$w_r$ is win ratio, $s_l$ is significant strikes landed per minute,
$s_a$ is significant strikes absorbed per minute, $a_s$ is striking accuracy,
$s_r$ is striking ratio, $d_t$ is average number of takedowns per 15 mintues,
$n_f$ is number of previous UFC fights, $n_w$ is number
of UFC wins, $c$ is the champion indicator variable, $e$ is college graduate.}
\label{other_features}
\end{table}
\end{center}

RF provides a way to gauge the importance or how dicriminatory
each feature is. The feature importances for a given feature
set are shown in Figure~\ref{rf_importances}. We see that age
and win ratio are the dominant features with leg reach and reach
also being relevant. We remind ourselves that because reach and leg
are highly correlated, we can think of them as also the same. Stance
plays only a small role which isn't surprising since most fights
take place between two orthodox fighters.

\begin{figure}[h]
\begin{center}
\includegraphics[width=11cm]{prediction/rf_importances.pdf}
\caption{Feature importances of the random forest model.}
\label{rf_importances}
\end{center}
\end{figure}

It is surprising that even with 121 features for each fight
and sophisicated machine learning models, we are only able to
predict 60\% of the fights correctly. The explanation for
the difficulty of prediction can be understood by assessing the
quality of the data. Figure~\ref{lack_of_ufc_fights} shows
a plot of the percentage of UFC fights where at least one of the
fighters had $n$ fights or fewer. We see that 70\% of UFC fights
have taken place between fighters with 3 or fewer previous UFC
fights. This means the data we have on them such as significant
strikes landed per minute and number of takedowns per 15 minutes
are not reliable. This explains why we are able to only improve on
our model based on age and reach using an addition 117 features.

\begin{figure}[h]
\begin{center}
\includegraphics[width=11cm]{prediction/lack_of_ufc_fights.pdf}
\caption{Percentage of fights between inexperienced UFC fighters.}
\label{lack_of_ufc_fights}
\end{center}
\end{figure}

We repeated our calculations using only fighters with
a certain number of fights.

\begin{center}
\begin{table}[h]
\begin{tabular}{ccr|ccc}
\toprule
Min. Number &        &         & RF        & Logistic Reg. & MLP      \\
of Fights & Fights & Features & Accuracy & Accuracy      & Accuracy\\
\hline
0  & 3300 & all 121 features & 58.1\% & 59.3\% & 59.1\% \\
2  & 1740 & all 121 features & 59.9\% & 58.4\% &  \\
4  & 1006 & all 121 features & 58.3\% & 59.8\% &   \\
4  & 1006 & $w_r$ & 52.5\% & 57.4\% & 57.4\%  \\
4  & 1006 & $w_r$, $a$ & 56.9\% & 58.9\% & 56.7\%  \\
4  & 1006 & $w_r$, $a$, $r$, $h$, $l$, $osw$ & 57.9\% & 58.6\% & 56.0\%\\
4  & 1006 & $w_r$, $a$, $s_l$, $s_a$ & 59.0\% & 60.0\% & 59.9\%  \\
4  & 1006 & $w_r$, $a$, $d_t$, $d_f$ & 57.4\% & 58.0\% & 56.1\%  \\
4  & 1006 & $w_r$, $a$, $s_r$ & 59.1\% & 60.8\% & 61.0\%  \\
4  & 1006 & $n_w$, $w_r$, $a$, $r$, $h$, $l$, $osw$, $c$, $e$, $s_r$, $s_l$, $s_a$, $d_t$ & 60.0\% & 59.8\% & 56.5\%  \\
6  &  586 & all 121 features & 59.9\% & 60.0\% & 54.5\% \\
8  &  343 & all 121 features & 58.5\% & 61.1\% & 54.0\% \\
\bottomrule
\end{tabular}
\caption{$d_f$ is average takedown defense}
\label{other_features_with_min}
\end{table}
\end{center}

To confirm this position, we have computed the career-average of
each fighters statistics and  used those as the features instead
of the moving averages, which we just argued are not very useful
since most fighters have so few fights. When these ``from the future''
values are used, we obtain the results
in Table~\ref{scores_using_career_stats}. Indeed, the career-averaged
values improve the performance of the model markedly. However, they
are off limits to us in reality. This tells us that fighters do
get better over time. To see this we plotted the squared deviation
versus normalized fight number in Figure~\ref{normalized_sl_dist}
for the 250 UFC fighters with the most fights. In the right panel
the average curve is shown. Indeed the curve relaxes to zero over
a non-trivial number of fights.

\begin{center}
\begin{table}[h]
\begin{tabular}{r|ccc}
\toprule
         & RF       & Logistic Reg. & MLP     \\
Features & Accuracy & Accuracy      & Accuracy\\
\hline
all 145 features & 66.0\% & 66.3\% & 66.0\% \\
8 career stats & 65.6\% & 66.3\% & 66.2\% \\
$\hat{s}_{l,\textrm{FM}}$, $\hat{s}_{a,\textrm{FM}}$ & 62.5\% & 64.4\% & 66.0\% \\
$\hat{s}_{l,\textrm{UFC}}$, $\hat{s}_{a,\textrm{UFC}}$ & 62.6\% & 64.6\% & 64.3\% \\
$s_l$, $s_a$ & 52.9\% & 56.8\% & 56.5\% \\
\bottomrule
\end{tabular}
\caption{Accuracy scores when future data is used
to compute the eight career statistics, which are then
used as features.
The eight career-averaged statistics are
significant strikes landed per minute,
significant striking accuracy,
significant strikes absorbed per minute,
significant strike defense  (the percentage of opponents strikes that did not land),
average takedowns landed per 15 minutes,
takedown accuracy,
takedown defense (the percentage of opponents TD attempts that did not land),
average submissions attempted per 15 minutes. $\hat{s}_{l,\textrm{FM}}$ is
the career-average value taken from FightMetric, which
includes fights outside the UFC, whereas $\hat{s}_{l,\textrm{UFC}}$ is
the career-average value computed using only UFC fights.
$s_l$ and $s_a$ are computed using only UFC data before the fight.}
\label{scores_using_career_stats}
\end{table}
\end{center}



\begin{figure}[h]
\begin{center}
\includegraphics[width=15cm]{prediction/sl_dist.pdf}
\caption{Feature importances of the random forest model.}
\label{normalized_sl_dist}
\end{center}
\end{figure}

One reason why predictions are so difficult is the lack of history
of the fighters. Of the 3300 fights, 892 fights one or both
fighters are making their debut. Because we ignore their previous
fights in other promotions we know nothing about them. Moreover,
70\% of the fights at least one of the fighters has three previous
UFC fights or less. The stats get screwed up if they don't have
a representative fights in the past.

The accuracy scores of various predictive models are found in Table~\ref{table_ML_models}.
We see the best model was found to be random forest with an accuracy of 56\%. This is
surprising given that the rule-based models in Table~\ref{table_simple_rules} had a
similar predictive power even though they were limited to a subset of the fights.
In the end the prediction is difficult because matchmakers try to pair fighters with
roughly equal ability.

What is the effect of guessing the initial values? How many fights have
at least one fighter new? Note that many quantities fluctuate dramatically
during the first few fights. This can cause problems. How many fights take
place between fighters with fewer than 5 fights?

Why does using the career statistics improve the model so much? What
happens if we only consider fights between fighters with a large number
of fights and we use the upto fight statistics? 





\appendix
\clearpage
\section{Scraping and Cleaning the Data}
\label{cleaning_data}
Data for this project was obtained from three different sources:
FightMetric.com, UFC.com and Wikipedia. Web scraping was used
in all cases by writing codes in Python which made use
of the Requests and BeautifulSoup modules. From FightMetric we scraped all the UFC fight
cards from UFC 1 to UFC 208. We also scraped the individual fighter
pages. From UFC.com we scraped the individual fighter pages which
give details on leg reach and education. We used to Wikipedia
to confirm or impute dates of birth, reach and height.

The FightMetric database of fighters has three pairs of fighters
with the same name: Michael McDonald, Dong Hyun Kim and
Tony Johnson. Each fighter name must be made unique since the name
serves as the primary key. One must also rename these fighters
in the FightMetric fight cards so that the two may be joined.
In the FightMetric fight cards there are two events with the same
name, namely, UFC Fight Night: Belfort vs Henderson. One should rename
one of the two events to group by event name. We also consolidated
event locations (e.g., ``Brooklyn, New York, USA'' became ``New York, New York, USA'').

The data obtained from UFC.com/fighter has duplicate names for
Gerard Gordeau and Dong Hyun Kim which were handled. Three fighters were
listed under the names ``... To be announced'', ``. To Be Determined'' and
``... To be determined'', which were removed. Some of the names had
two spaces between the first and last names. The fighter Matt Schnell
is listed as having a 72 inch leg reach which must be false.
After the cleaning there were 1897 fighters of which 533 were active. Note that
the UFC data set contains WEC fighters and maybe those from other promotions
acquired by the UFC.

The FightMetric and UFC reach data are in perfect agreement for 824 fighters.
They differ by 1 inch for 19 fighters and 2 inches for 18. Only 5 fighters
differ by more than 2 inches.
With respect to height, the two data sets agree perfectly for 1351 fighters,
differ by 1 inch for 131 fighters and 2 inches for 36. This leaves 19
cases where the height difference is greather than 2 inches. Dates of birth
are harder to check because the UFC only gives age.
Overall the agreement is quite good with maybe only 3\% of cases not
agreeing perfectly.
The author contacted FightMetric to ask about the discrepancies. Their reply
was that they believe their data to be the best source.

From UFC 1 to UFC 208 there have been 4096 fights between 1641 unique fighters. The
data obtained by scraping UFC.com is missing only 13 of these 1641. The missing
fighters are Benji Radach, Danillo Villefort, Eddie Mendez, Edilberto de Oliveira,
Joao Pierini, Julian Sanchez, Kit Cope, Luiz Cane, Masutatsu Yano, Nate Loughran,
Saeed Hosseini, Scott Smith and Tito Ortiz. Note that fuzzy matching was useful
in matching many of the names between the FightMetric and UFC data sets.



\section{UFC Fighters with College Experience}
\label{appendix_education}
\begin{center}
\input{education/education_sample.tex}
\end{center}

%\clearpage
%\section{Every UFC Fight up to UFC 208}
%\label{every_ufc_fight}
%\begin{center}
%\input{../fights_table.tex}
%\end{center}

\end{document}
