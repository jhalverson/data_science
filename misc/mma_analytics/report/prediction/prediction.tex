\clearpage
\section{Predicting the Winner}

In this section we create several models to predict
the winner. We start with simple models based on rules
and then move on to machine learning models such as
random forests, artificial neural networks and
adaptive boosting.

\subsection*{Simple Rules}

If we exclude no contests and draws, since January 1, 2005,
there have been 3561 UFC fights with a winner and a loser.
Table~\ref{table_simple_rules} shows the accuracy score
for several simple rule-based models. For instance, if
our model is that the younger fighter always wins, we would
be correct 1964 times out of 3529 fights or 55.7\%.
There are 32 fights where either the dates of birth
are unknown or both fighters have the same date of birth so
these are excluded in the calculation.

\begin{center}
\begin{table}[h]
\begin{tabular}{r|ccccc}
  \toprule
  Model & Accuracy & Wins & Total & Excluded & $\log_{10}p$\\
  \hline
  Lesser age wins & 55.7\% & 1964 & 3529 & 32 & $-11$\\
  Lesser age by 4 years or more wins & 59.1\% & 893 & 1511 & 2050 & $-12$\\
  Greater reach wins & 52.4\% & 1499 & 2863 & 698 & $-2$\\
  Greater reach by 4 inches or more wins & 55.5\% & 523 & 942 & 2619 & $-4$\\
  Greater height wins & 51.2\% & 1486 & 2903 & 658 & $-0.7$\\
  Greater leg reach wins & 49.0\% & 487 & 993 & 2568 & $-0.2$\\
  Southpaw (vs. orthodox) wins & 55.5\% & 561 & 1010 & 2551 & $-4$\\
  College graduate (vs. non-graduate) wins & 56.5\% & 594 & 1051 & 2510 & $-5$\\
  Greater number of UFC fights wins & 55.6\% & 1606 & 2886 & 675 & $-9$\\
  Fighter who fought more recently wins & 51.8\% & 1220 & 2353 & 1208 & $-1.1$\\
  Greater previous win ratio wins & 59.6\% & 439 & 737 & 2824 & $-7$\\
  Current or former champion wins & 61.8\% & 175 & 283 & 3278 & $-5$\\
  \hline
  Lesser age and greater reach wins & 57.2\% & 919 & 1607 & 1954 & $-9$\\
  Southpaw and lesser age wins & 61.7\% & 284 & 460 & 3101 & $-7$\\
  Southpaw, lesser age and greater reach wins & 62.0\% & 132 & 213 & 3348 & $-4$\\
  Greater previous win ratio and lesser age wins & 62.9\% & 256 & 407 & 3154 & $-7$\\
  \bottomrule
\end{tabular}
\caption{Accuracy scores for simple rule-based models. For models where win ratio was used, we only considered
fighters with 5 or more UFC fights. Current or former champion refers to fights
between belt holders (current, interim or former) versus those who have never held a
UFC title. When southpaw is part of the model, only fights between southpaws are
orthodox fighters are considered. $p$-values are given in the rightmost column. Note that for $p=0.05$, $\log_{10}p = -1.3$. In all
cases the sum of 
Total and Excluded is 3561. Random guessing corresponds to an accuracy of 50\%.}
\label{table_simple_rules}
\end{table}
\end{center}

In computing the number of previous UFC fights (and wins) for each fighter, we start
counting with UFC 1. We ignored no contests but draws were included. A fighter's
win ratio is their number of UFC wins divided by their
total number of UFC fights (including draws).
Table~\ref{table_simple_rules} makes clear the advantage of being young, having
greater reach and being a southpaw. Note that Conor McGregor has all three of
these traits.

Before moving on to machine learning models, we explore two simple
models. The first is the Greater Win ratio Wins (GWW) model,
which predicts the fighter with the greater win ratio to win. The
second is the Win ratio-Age-Reach (WAR) model which works as follows:

\begin{enumerate}[noitemsep]
\item If the win ratios are unequal then the fighter with the higher win ratio wins
\item If the win ratios are equal then the younger fighter wins
\item If win ratios and ages are equal then the fighter with greater reach wins
\item If all three quantities are equal then randomly guess
\end{enumerate}

\noindent
The advantage of the WAR model is that it can be applied to any fight
since it always makes a prediction whereas the GWW model can only be applied
to fights where the win ratios are unequal. We use dates of birth with the
WAR model so usually a prediction is made before reach is considered.

Before applying these models we must address the choice of the win
ratio for a fighter making their UFC debut.
Using the 3561 fights since January 1, 2005, there are
299 fights where both fighters are debuting and 711 where only one
of the fighters is debuting. Therefore, 1010 out of the 3561 fights or
28.4\% have at least one fighter debuting.
The win ratio of a fighter making their debut is
set to zero for the calculations described below.

In Table~\ref{table_simple_rules} when using the win ratio we only
considered fights between fighters who had at least 5 previous UFC bouts each.
Here we consider the general case.
The top panel of Figure~\ref{win_ratio_wins} shows the accuracy
of the two models as a
function of the minimum number of fights, $n_{\textrm{min}}$.
The number of fights which meet the condition
set by $n_{\textrm{min}}$ are shown in the
lower panel.
The average accuracy of both the GWW and WAR models is 59.7\% when
$n_{\textrm{min}}$ ranges from 3 to 10. As $n_{\textrm{min}}$ becomes
large the number of fights becomes small and both models
become unreliable. The number of fights first drops below 100 in
going from $n_{\textrm{min}}=11$ to 12. When $n_{\textrm{min}}=0$,
we see in the lower panel that while the WAR model makes
a prediction for all 3561 fights, the GWW model only makes predictions for 2699 fights
since it can only be applied to fights where the win ratios are unequal.

\begin{figure}[h]
\begin{center}
\includegraphics[width=9cm]{prediction/win_ratio_model.pdf}
\caption{Accuracy and number of fights used in the calculation
versus minimum number of previous fights, $n_{\textrm{min}}$, required for each fighter.}
\label{win_ratio_wins}
\end{center}
\end{figure}

\clearpage
\subsection*{Machine Learning and Feature Engineering}

One limitation of simple rule-based models is that they
cannot be applied to all fights. For instance, if the two fighters
have the same reach, a model that predicts the fighter with the longer
reach to win must exclude the fight.
In this section we go beyond simple rules to machine learning,
which is a branch of Artificial Intelligence.

In the first stage of supervised machine learning, a model
is trained on information about the two fighters
and which of the two won the bout. After the training phase,
the model is said to be smart and we show it information about
two fighters and ask it to tell us who won.

For each fighter we have the following features:

\begin{itemize}[noitemsep]
  \item Height, $h$
  \item Reach, $r$
  \item Leg reach, $l$
  \item Stance (orthodox, southpaw or switch), $osw$
  \item Date of birth or age, $a$
  \item College graduate, $e$
  \item Current or former champion/interim champion, $c$
  \item Number of UFC fights, $n_f$
  \item Number of UFC wins, $n_w$
  \item Win ratio of UFC fights, $w_r=n_w/n_f$
  \item Total time in the Octogon, $t_T$
  \item Elapsed time since last UFC fight, $t_e$
  \item Fraction of wins by KO/TKO, $W_{\textrm{TKO}}$
  \item Fraction of wins by submission, $W_{\textrm{SUB}}$
  \item Fraction of wins by decision, $W_{\textrm{DEC}}$
  \item Fraction of losses by KO/TKO, $L_{\textrm{TKO}}$
  \item Fraction of losses by submission, $L_{\textrm{SUB}}$
\end{itemize}

\noindent
We also use the following offensive and defensive metrics:

\begin{itemize}[noitemsep]
  \item Significant strikes landed per minute, $s_l$
  \item Significant striking accuracy, $a_s$
  \item Significant strikes absorbed per minute, $s_a$
  \item Significant striking defense, $a_f$
  \item Average takedowns landed per 15 minutes, $d_t$
  \item Takedown accuracy, $d_a$
  \item Takedown defense, $d_f$
  \item Average submissions attempted per 15 minutes, $s_b$
\end{itemize}

\begin{itemize}[noitemsep]
  \item Significant strikes attempted per minute
  \item Knockdowns per minute
  \item Total strikes landed per minute
  \item Total strikes attempted per minute
  \item Total strikes absorbed per minute
  \item Average takedowns attempted per 15 minutes
  \item Average passes per 15 minutes (e.g., half-mount to side-mount)
  \item Average reversals per 15 minutes (e.g., half-guard to full-guard)
\end{itemize}

\noindent
The derived features are

\begin{itemize}[noitemsep]
  \item Difference in age, reach and so on of the two fighters (e.g., $\Delta r=r_1-r_2$)
  \item Total damage absorbed = cumulative sum of significant strikes absorbed
  \item Striking ratio, $s_r =$ significant strikes landed / significant strikes absorbed
  \item Natural logarithm of elapsed time since last UFC fight, $\ln t_e$
  \item Fighter versatility index
\end{itemize}

\noindent
The fighter versatility index was introduced by Estelami as
\begin{equation}
I = \frac{\frac{1/3}{W_{\textrm{TKO}}^2+W_{\textrm{SUB}}^2+W_{\textrm{DEC}}^2}-\frac{1}{3}}{2/3}.
\end{equation}

\noindent
The fighter versatility index is zero when a fighter achieves all wins in one manner and
it is unity when all three methods of victory are uniformly employed.\\

\noindent
The following additional features depend on both fighters:

\begin{itemize}[noitemsep]
  \item Expected significant strikes absorbed per minute of fighter 1 = $a_{f,1} (s_{l,2}/a_{s,2})$
  \item TKO susceptibility of fighter 1 = $(L_{\textrm{TKO},1} \times W_{\textrm{TKO},2})^{1/2}$
  \item Submission susceptibility of fighter 1 = $(L_{\textrm{SUB},1} \times W_{\textrm{SUB},2})^{1/2}$
  \item Same three above except for fighter 2
\end{itemize}

It is important to note that all the quantities described above are computed
up to the time of the fight. That is, we do not use any information
obtained during the fight or afterwards to make predictions.
Later, we will violate this rule to see if such an approach
leads to an improvement in model performance.

What values should we use for a fighter making their UFC debut? In this work
we set all their values involving wins and losses to
zero (e.g., $w_r=W_{\textrm{TKO}}=I=0$, etc.). However, for
offensive and defensive metrics such as striking accuracy we use the UFC median values.
Alternative approaches might set these values to zero, impute values based
on age, reach, stance and weight class, or even acquire data from
each fighter's previous promotions and use those
values.

The most important part of any machine learning approach is
to assemble a collection of discriminatory features. While the
list above seems quite comprehensive, much more could be done.
Some of the features are highly correlated and
we will address this in the next section.


\subsection*{Prediction using Machine Learning Models}

In cases where height and reach were known but leg reach
was missing, we imputed values by creating a linear model
based on reach and height. A second model was
used to impute leg reach when only height was known. After removing fights where
dates of birth were absent, we were left with 3300 fights,
all of which took place after January 1, 2005, which was
the ``modern era''cutoff that we introduced. However,
keep in mind that quantities such as number of UFC fights,
win ratio and significant strikes landed per minute were computed
using all UFC fights. We simply train the model and make
predictions on fights that took place after the cutoff but
use data before it.

Before creating our first machine learning model we examine a correlation
matrix of certain features. Table~\ref{corr_mat} shows that
most correlation coefficients are small. The largest ones
are for reach and height ($r=0.88$). This suggests that training a model
on both of these features will not be useful and may even be
detrimental. We see that there is a positive
correlation between win ratio and significant strikes landed
per minute while the correlation is negative between win ratio
and significant strikes absorbed per minute. Both of these trends
are logical and serve as a check.

\begin{center}
\begin{table}[h]
\input{prediction/corr_table.tex}
\caption{Correlation matrix for basic features.}
\label{corr_mat}
\end{table}
\end{center}

During the training phase, we randomly select 70\% of the fights
and use this data to fit the models. This is done using
stratified K-fold cross validation with 10 folds. A grid search
procedure is used to find the optimal hyperparameters for each
model. The optimized model is then used to predict the outcome
of the remaining 30\% of the data. The accuracy is then determined
by comparing the predicted outcomes with the known outcomes.
Because results vary with train-test split, we shuffle the
data and repeat the procedure above ten times. The average
of these ten accuracy scores are then reported.
While using the
test data more than once is considered bad practice, this approach eliminates
concerns associated with using a single split which may give
favorable or unfavorable results depending on how the data
happens to be partitioned.

Random forests (RF) is a tree-based ensemble method,
logistic regression (LR) is an application of the generalized
linear model with the logit link function, the multilayer perceptron (MLP)
is an artificial neural network and AdaBoost is
an ensemble method based on adaptive boosting.
For each model we optimize the hyperparameters
during cross-validation. For RF the
number of estimators is 100, bootstrapping is used
and the number of features to split on is taken
as the square root of the total number of features.
For LR we optimize the regularization coefficient
as well as the type of regularization ($L_1$ or $L_2$).
The sizes of the hidden layers were set to (5, 2) for
the MLP and $\alpha$, the $L_2$ regularization parameter, was optimized.
Lastly, for AdaBoost, the depth of the decision trees
(base classifiers), the learning rate and the number
of estimators were optimized.
For non-tree based methods, it is necessary to standardize
each feature. The standardizer was fit to the train data
and then used to transform both the train and test data.

\subsection*{Anthropometric Models}

Our first set of models is based on age, height,
reach, leg reach and stance. The accuracy scores
for different models and feature sets are show
in Table~\ref{anthropomorphic_features}.
We see that when age is the only feature, LR is gives
an accuracy of 55.6\%. This is very close to the result obtained using the simple
rule-based model in the previous section. The table
suggests that the best we can do is just more than
56\%. Given the strong correlation between height, reach
and leg reach, it is not surprising that results do not
improve when these features are used together. We see that
age and reach are the leading discriminatory features of the set.

\begin{center}
\begin{table}[h]
\begin{tabular}{r|cccc}
  \toprule
   & Random Forest & Logistic Reg. & Multilayer Perceptron \\
  Features & Accuracy & Accuracy& Accuracy \\
  \hline
  $a$ & 53.4\% & 55.6\% & 55.5\%\\
  $r$ & 52.2\% & 52.1\% & 51.9\%\\
  $a$, $r$ & 52.7\% & 56.8\% & 56.1\%\\
  $\Delta a$, $\Delta r$ & 52.9\% & 56.3\% & 56.2\%\\
  $h$, $r$ & 52.9\%  & 52.1\% & \\
  $\Delta h$, $\Delta r$ & 51.6\% & 51.6\%  &\\
  $h$, $\Delta h$, $r$, $\Delta r$ & 52.0\% & 51.9\%  &\\
  $h$, $r$, $l$ & 56.6\% & 51.9\% &  \\
  $\Delta h$, $\Delta r$, $\Delta l$ & 50.7\% & 52.2\%  &\\
  $\Delta h$, $\Delta r$, $\Delta l$, $\Delta a$ & 52.4\% & 55.1\% & 55.4\% \\
  $h$, $r$, $a$ & 53.4\% & 54.7\% &  \\
  $h$, $r$, $a$, $a^2$ & 53.4\% & 55.1\% &  \\
  $r/h$, $a^2$ & 54.3\% & 55.1\% & 55.2\%  \\
  $h$, $r$, $l$, $a$ & 56.1\% & 54.9\% &  \\
  $h$, $r$, $l$, $a$, $\Delta h$, $\Delta r$, $\Delta l$, $\Delta a$ & 55.8\% & 54.9\%  &\\
  $osw$ & 52.3\% & 52.3\% & 52.3\%\\
  $a$, $osw$ & 51.0\% & 55.0\% & \\
  $r$, $a$, $osw$ & 53.5\% & 55.7\% &  \\
  $r$, $l$, $a$, $osw$ & 56.1\% & 55.4\% & \\
  $h$, $r$, $l$, $a$, $osw$ & 56.7\% & 55.3\% &  \\
  $h$, $r$, $l$, $a$, $osw$, $\Delta h$, $\Delta r$, $\Delta l$, $\Delta a$, $\Delta osw$ & 56.0\% & 54.8\% &  \\
  \bottomrule
\end{tabular}
\caption{Accuracy scores for different machine learning classifiers which were trained
on age, stance and anthropometric measurements only.
$a$ is age, $h$ is height, $r$ is reach, $l$ is leg reach,
$osw$ stands for three indicator variables for orthodox, southpaw,
switch and $\Delta$ is the difference in one of the features between the two
fighters. In the third row the models were trained on
four features or the age and reach of both fighters (i.e., $a_1$, $r_1$, $a_2$, $r_2$)
whereas in the fourth row only two features were used (i.e., $a_1-a_2$ and $r_1-r_2$).}
\label{anthropomorphic_features}
\end{table}
\end{center}

\clearpage
\subsection*{Full Feature Models}

Table~\ref{other_features} gives results for models with more
advanced choices of the features. We see that LR with win ratio
and age as the features (i.e., $w_{r,1}$, $w_{r,2}$, $a_1$, $a_2$) gives an
accuracy score of 57.5\%. As more features are introduced such as
total number of fights, stance, champion status and education, the
accuracy increases somewhat. However, we only achieve 60\% when the feature
extraction technique called linear discriminant analysis is applied
to the full feature matrix and then LR is used. Overall the the accuracy
scores are surprisingly low.

\begin{center}
\begin{table}[h]
\begin{tabular}{r|cccc}
\toprule
         & RF       & Logistic Reg. & MLP      & AdaBoost\\
Features & Accuracy & Accuracy      & Accuracy & Accuracy\\
\hline
$w_r$ & 54.0\% & 55.0\% & 55.9\% & \\
$\Delta w_r$ & 52.1\% & 55.5\% & 55.5\% & \\
$w_r$, $n_f$ & 52.7\% & 55.1\% & 53.7\% & \\
$w_r$, $n_f$, $n_w$ & 53.4\% & 55.8\% &  & \\
$w_r$, $a$ & 55.5\% & 57.5\% & 56.6\% & \\
$\Delta w_r$, $\Delta a$ & 53.3\% & 57.6\% & 57.3\% & \\
$n_f$, $n_w$, $a$, $r$ & 55.1\% & 57.0\% \\
$n_f$, $n_w$, $a$, $r$, $h$, $l$, $c$ & 57.8\% & 58.0\% & &\\
$n_f$, $n_w$, $a$, $r$, $h$, $l$, $osw$ & 57.2\% & 58.8\% & & 58.1\%\\
$n_f$, $n_w$, $a$, $r$, $h$, $l$, $osw$, $c$, $e$ & 58.0\% & 58.8\% & &\\
$n_f$, $n_w$, $a$, $r$, $h$, $l$, $osw$, $c$, $e$, $s_a$ & 58.0\% & 58.5\% & & \\
$n_f$, $n_w$, $a$, $r$, $h$, $l$, $osw$, $c$, $e$, $s_a$, $s_l$ & 57.9\% &59.1\% &58.9\% & \\
$n_f$, $n_w$, $a$, $r$, $h$, $l$, $osw$, $c$, $e$, $a_s$ & 58.6\% & 58.4\% \\
$n_f$, $n_w$, $a$, $r$, $h$, $l$, $osw$, $c$, $e$, $\Delta s_r$ & 57.8\% & 58.2\% & 58.4\% & 58.1\%\\
$n_f$, $n_w$, $a$, $r$, $osw$, $c$ & 56.6\% & 58.9\% & 58.5\% \\
$n_f$, $n_w$, $a$, $r$, $osw$, $c$, $h$, $l$ & 57.9\% & 59.0\% & &\\
$a$, $r$, $s_l$ & 54.9\% & 56.2\% & & \\
$a$, $r$, $d_t$ & 54.3\% & 56.1\% & & \\
$a$, $r$, $s_a$ & 55.7\% & 57.1\% & & \\
$s_l$, $s_a$ & 52.9\% & 56.8\% & 56.5\% & \\
all 121 features & 58.1\% & 59.3\% & 59.1\% & \\
all 121 features + LDA &  & 60.1\% & & \\
\bottomrule
\end{tabular}
\caption{$w_r$ is win ratio, $s_l$ is significant strikes landed per minute,
$s_a$ is significant strikes absorbed per minute, $a_s$ is striking accuracy,
$s_r$ is striking ratio, $d_t$ is average number of takedowns per 15 minutes,
$n_f$ is number of previous UFC fights, $n_w$ is number
of UFC wins, $s_r$ is striking ratio, $c$ is the champion indicator variable and $e$ is college graduate.}
\label{other_features}
\end{table}
\end{center}

RF provides a way to gauge the importance of
each feature. The importances for a given feature
set are shown in Figure~\ref{rf_importances}. We see that age
and win ratio are the dominant features with leg reach and reach
also being relevant. We remind ourselves that because reach and leg reach
are highly correlated, RF could probably use them interchangeably. Stance
plays only a small role which is not surprising since most fights
take place between two orthodox fighters.

\begin{figure}[h]
\begin{center}
\includegraphics[width=11cm]{prediction/rf_importances.pdf}
\caption{Feature importances of the random forest model with $w_r$,
$a$, $r$, $h$, $l$ and $osw$.}
\label{rf_importances}
\end{center}
\end{figure}

It is surprising that even with a large number of features
and robust machine learning models, we are only able to
predict 60\% of the fights correctly. 
The difficulty of prediction can partly be understood by assessing the
quality of the data. Figure~\ref{lack_of_ufc_fights} shows
a plot of the percentage of UFC fights where at least one of the
fighters had $m$ previous UFC fights or less. We see that 71\% of UFC fights
have taken place between fighters where at least one of the two
fighters had 3 or fewer previous UFC
bouts. Furthermore, in 28\% of the fights one or both
fighters were making their debut.
Note that while we only considered fights after January 1, 2005 (orange curve),
we computed the number of previous fights using all the data. Because
so many fights take place between fighters with so few UFC fights,
metrics such as significant
strikes landed per minute and number of takedowns per 15 minutes
are not reliable in many cases.
This partly explains why we were only able to only improve on
our simple models of the previous section by a few percent.

Given the large number of fights where one or both fighters
are debuting, it is important to initialize the features properly.
What value should we use for the takedown accuracy, for example, of a
fighter who has never fought in the UFC?
The overall accuracy of the model
will clearly be sensitive to these choices. It is likely that
using a second model specifically for handling debuts would improve
the overall accuracy. However, this is beyond the scope of this
work.

\begin{figure}[h]
\begin{center}
\includegraphics[width=11cm]{prediction/lack_of_ufc_fights.pdf}
\caption{Percentage of UFC fights where one or both fighters
have only $m$ previous UFC fights or less. For the
`Since 2005' data set with $m=0$, 1, 2, 3, 4, 5 the values are
28.0\%, 48.8\%, 61.5\%, 70.9\%, 77.7\% and 83.1\%, respectively.
For the construction of this figure we include draws and 
no contest outcomes.}
\label{lack_of_ufc_fights}
\end{center}
\end{figure}

To verify the hypothesis that fights between inexperienced UFC fighters were
lowering our accuracy, we retrained our models using
bouts where both fighters had a minimum number of previous
UFC fights (see Table~\ref{other_features_with_min}).
This indeed resulted in improved performance. For example,
using $w_r$, $a$ and $s_r$, the LR accuracy improved from 57.3\% to 60.8\% when the minimum
number of fights was raised from zero to four.
While the improvement is still only slight, it lends support
to our hypothesis. As the minimum number of fights increases,
the number of fights that meet this condition decreases. In fact,
from UFC 1 to UFC 208, if we include draws and no contests, there are
only 364 fights out of 4068 or 8.9\% where both fighters have 8 previous UFC
fights or more. This fact emphasizes the difficulty
of making predictions based on UFC data only. Presumably if we incorporated
data from Strikeforce, WEC, Bellator FC, Pride FC and so on, our predictions would
be better.

\begin{center}
\begin{table}[h]
\begin{tabular}{ccr|ccc}
\toprule
Min. Number &        &         & RF        & Logistic Reg. & MLP      \\
of Fights & Fights & Features & Accuracy & Accuracy      & Accuracy\\
\hline
0  & 3300 & $w_r$ & 54.0\% & 55.0\% & 55.9\%  \\
0  & 3300 & $w_r$, $a$, $s_r$ & 55.2\% & 57.3\% & 58.0\%  \\
0  & 3300 & $w_r$, $a$, $r$, $s_r$ & 57.0\% & 58.1\% & 58.5\%  \\
0  & 3300 & all 121 features & 58.1\% & 59.3\% & 59.1\% \\
4  & 1006 & $w_r$ & 52.5\% & 57.4\% & 57.4\%  \\
4  & 1006 & $w_r$, $a$, $s_r$ & 59.1\% & 60.8\% & 61.0\%  \\
4  & 1006 & $w_r$, $a$, $r$, $s_r$ & 57.9\% & 60.5\% & 58.5\% \\
4  & 1006 & all 121 features & 58.3\% & 59.8\% &   \\
4  & 1006 & $w_r$, $a$ & 56.9\% & 58.9\% & 56.7\%  \\
4  & 1006 & $w_r$, $a$, $r$, $h$, $l$, $osw$ & 57.9\% & 58.6\% & 56.0\%\\
4  & 1006 & $w_r$, $a$, $s_l$, $s_a$ & 59.0\% & 60.0\% & 59.9\%  \\
4  & 1006 & $w_r$, $a$, $d_t$, $d_f$ & 57.4\% & 58.0\% & 56.1\%  \\
4  & 1006 & $n_w$, $w_r$, $a$, $r$, $h$, $l$, $osw$, $c$, $e$, $s_r$, $s_l$, $s_a$, $d_t$ & 60.0\% & 59.8\% & 56.5\%  \\
6  &  586 & all 121 features & 59.9\% & 60.0\% & 54.5\% \\
8  &  343 & all 121 features & 58.5\% & 61.1\% & 54.0\% \\
\bottomrule
\end{tabular}
\caption{Model performance when only fights with experienced
fighters are considered. The leftmost column specifies how
many previous UFC fights each fighter must have in order
for the fight to be included in the calculation. The second
column from the left indicates the number of fights out of 3300 that
meet that condition. $d_f$ is average takedown defense. Other symbols
have been defined previously.}
\label{other_features_with_min}
\end{table}
\end{center}

\clearpage
\subsection*{Using Data from the Future}

In the previous section we argued that if we had more
well-established values for metrics such as significant strikes
landed per minute and takedown defense, we could do a better
job with prediction. We then retrained our models on
fights between experienced UFC fighters
and indeed the accuracy was higher. In this section we use
career-average values for each fighter as the features.
For example, to compute the career-average value of significant
strikes landed per minute, $\hat{s}_l$, we add all the significant strikes
the fighter has landed in their UFC career and then divide by the
total time spent in the Octogon. This
is cheating because we are using data from the future but we
will see that it is an instructive exercise.

When the career-average values are used, we obtain the results
in Table~\ref{scores_using_career_stats}. Indeed, these
values improve the performance of the model markedly.
For instance, using only $\hat{s}_l$ and $\hat{s}_a$ we achieve an accuracy of
64.4\% versus 56.8\% when these quantities are only
computed up to the time of the fight.

\begin{center}
\begin{table}[h]
\begin{tabular}{r|ccc}
\toprule
         & RF       & Logistic Reg. & MLP     \\
Features & Accuracy & Accuracy      & Accuracy\\
\hline
all 145 features & 66.0\% & 66.3\% & 66.0\% \\
8 career metrics & 65.6\% & 66.3\% & 66.2\% \\
$\hat{s}_{l,\textrm{FM}}$, $\hat{s}_{a,\textrm{FM}}$ & 62.5\% & 64.4\% & 66.0\% \\
$\hat{s}_{l,\textrm{UFC}}$, $\hat{s}_{a,\textrm{UFC}}$ & 62.6\% & 64.6\% & 64.3\% \\
$s_l$, $s_a$ & 52.9\% & 56.8\% & 56.5\% \\
$\hat{w}_r$ & 75.7\% & 79.6\% & 79.4\% \\
$w_r$ & 54.0\% & 55.0\% & 55.9\% \\
\bottomrule
\end{tabular}
\caption{Accuracy scores when future data is used
to compute the eight career statistics, which are then
used as features.
The eight career-average metrics are
significant strikes landed per minute,
significant striking accuracy,
significant strikes absorbed per minute,
significant strike defense  (the percentage of opponents strikes that did not land),
average takedowns landed per 15 minutes,
takedown accuracy,
takedown defense (the percentage of opponents TD attempts that did not land),
average submissions attempted per 15 minutes. $\hat{s}_{l,\textrm{FM}}$ is
the career-average value of significant strikes landed per minute
taken from FightMetric, which
includes fights outside the UFC, whereas $\hat{s}_{l,\textrm{UFC}}$ is
the career-average value computed using only UFC fights.
$s_l$ and $s_a$ are computed using only UFC data before the fight. $\hat{w}_r$
is the career-average win ratio.}
\label{scores_using_career_stats}
\end{table}
\end{center}

This result begs the question how many fights are needed for metrics such
as significant strikes landed per minute
to become reasonably well-established (i.e., $\bar{s}_l \approx \hat{s}_l$)?
To answer this question let's first consider a specific fighter.
Table~\ref{anderson_running_ave} shows these values for
Anderson Silva. In his first fight against Leben, he landed
17 significant strikes in 48 seconds to finish his opponent.
This gave Anderson a $s_l$ value of 20.82. His next few fights took
longer and after his fight with Griffin he was landing
4.2 significant strikes per minute. Anderson's career-average
value is 3.01 and we see he reached this value roughly around his
12th fight.

\begin{center}
\begin{table}[h]
\input{prediction/anderson_slpm.tex}
\caption{Anderson Silva's UFC fights ignoring his no contest with Nick Diaz. $s_L$ is the number
of significant strikes landed per fight by Silva, $\Sigma s_L$ is the cumulative
sum of $s_L$, $\bar{s}_l$ is the moving average of $s_L/\textrm{Minutes}$,
$\hat{s}_l$ is Silva's career-average value and $n_f=21$.}
\label{anderson_running_ave}
\end{table}
\end{center}

The left panel of Figure~\ref{normalized_sl_dist} shows the approach of
$\bar{s}_l$ to its career-average value for Anderson Silva and 247 other
UFC fighters who have 9 UFC fights or more. The horizontal
scale is the fight number normalized by the total number of fights for
each fighter. We see that during the early stages of a fighter's career, $\bar{s}_l$ deviates
significantly from $\hat{s}_l$. At later stages $\bar{s}_l$
approaches $\hat{s}_l$ and they become equal at $n/n_f=1$.
In the right panel we show the average of the 248 curves
on the left. The average curve is sufficiently close to zero at $n/n_f \approx 1/2$.
The inset shows a histogram of the total number of fights of the 248 fighters. The
mean value of $n_f$ is about 14.
Thus,
the answer to our original question is that $\bar{s}_l$ becomes
well-established around
$n/n_f=1/2$ which corresponds to about 7 fights on average. This is supported
by the last entry in Table~\ref{other_features_with_min} where we achieve our highest
accuracy of 61.1\% when only past data is used and 
only fights where both fighters have
had 8 UFC bouts or more are considered.

\begin{figure}[h]
\begin{center}
\includegraphics[width=15cm]{prediction/sl_dist.pdf}
\caption{(left) Square of the difference between the moving average
and overall average of significant strikes landed per minute as
a function of normalized fight number for all fighters with
9 UFC fights or more (248 fighters). $n$ is the $n$-th UFC fight of
a given fighter and $n_f$ is their total number of UFC fights. Hence,
$n$ ranges from 1 to $n_f$.
Note that $\bar{s}_l=\hat{s}_l$ at $n/n_f=1$ for all fighters.
(right) Same as left panel except averaged over the curves in the left.
The inset shows a histogram of total number of fights for the
fighters considered in the left panel.}
\label{normalized_sl_dist}
\end{center}
\end{figure}

In summary, one problem we face with prediction is that the vast
majority of UFC fights take place between fighters with few
previous UFC bouts. This means the metrics or features for these
fighters are not well-established since we showed that roughly 7 fights are needed
before they become reliable (at least for $s_l$). However, even when fighters have
a sufficient number of fights, our best model still only gives
61.1\% accuracy when each fighter has 8 fights or more. To improve on this one must develop features
that are more discriminatory. One way to do this is to recognize that
not all wins and losses are same. Losing to a future longtime champion
should mean less than a loss to an average fighter. Similarly, scoring
a takedown against a fighter with the best takedown defense should receive
more weight. The same goes for striking offense and defense. Also, more data is available from FightMetric. For instance,
time-in-position (TIP) data would be very useful. 
TIP data tells how much of the fight took place at distance,
in the clinch and on the ground.
If a fight
took place almost entirely on the ground, it doesn't make sense to compute the average
number of takedowns per minute by using the total fight time. The present
study ignores this fact.
Presumably, TIP data and other information could be
used to create derived features which would lead to better
accuracy.
