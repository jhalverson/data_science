\clearpage
\section{Predicting the Winner}

In this section we create several models to try to predict
the winner. We start with simple models based on rules
and then move on to machine learning models such as
random forests, artificial neural networks and
adaptive boosting. It will be shown that prediction is quite
difficult with our best model having an accuracy of
only about 60\%.

\subsection*{Simple Rules}

If we exclude no contests and draws, since January 1, 2005,
there have been 3561 UFC fights with a winner and a loser.
Table~\ref{table_simple_rules} shows the accuracy score
for several simple rule-based models. For instance, if
our model is that the younger fighter always wins, we would
be correct 1964 times out of 3529 fights or 55.7\%.
There are 32 fights where either the dates of birth
are unknown or both fighters have the same date of birth so
these are excluded in the calculation.

\begin{center}
\begin{table}[h]
\begin{tabular}{r|ccccc}
  \toprule
  Model & Accuracy & Wins & Total & Excluded & $\log_{10}p$\\
  \hline
  Lesser age wins & 55.7\% & 1964 & 3529 & 32 & $-11$\\
  Lesser age by 4 years or more wins & 59.1\% & 893 & 1511 & 2050 & $-12$\\
  Greater reach wins & 52.4\% & 1499 & 2863 & 698 & $-2$\\
  Greater reach by 4 inches or more wins & 55.5\% & 523 & 942 & 2619 & $-4$\\
  Greater height wins & 51.2\% & 1486 & 2903 & 658 & $-0.7$\\
  Greater leg reach wins & 49.0\% & 487 & 993 & 2568 & $-0.2$\\
  Southpaw (vs. orthodox) wins & 55.5\% & 561 & 1010 & 2551 & $-4$\\
  Education (vs. no education) wins & 56.5\% & 594 & 1051 & 2510 & $-5$\\
  Greater number of UFC fights wins & 55.6\% & 1606 & 2886 & 675 & $-9$\\
  Fighter who fought more recently wins & 51.8\% & 1220 & 2353 & 1208 & $-1.1$\\
  Greater previous win ratio wins & 58.6\% & 434 & 740 & 2821 & $-6$\\
  Current or former champion wins & 61.8\% & 175 & 283 & 3278 & $-5$\\
  \hline
  Lesser age and greater reach wins & 57.2\% & 919 & 1607 & 1954 & $-9$\\
  Southpaw and lesser age wins & 61.7\% & 284 & 460 & 3101 & $-7$\\
  Southpaw, lesser age and greater reach wins & 62.0\% & 132 & 213 & 3348 & $-4$\\
  Greater previous win ratio and lesser age wins & 61.9\% & 252 & 407 & 3154 & $-6$\\
  \bottomrule
\end{tabular}
\caption{Accuracy scores for simple rule-based models. For models where win ratio was used, we only considered
fighters with 5 or more UFC fights. $p$-values are given in the rightmost column. Note that for $p=0.05$, $\log_{10}p = -1.3$. In all
cases the sum of 
Total and Excluded is 3561.}
\label{table_simple_rules}
\end{table}
\end{center}

In computing the number of previous UFC fights for each fighter, we start
counting with UFC 1. We ignored no contests but draws were included. A fighter's
win ratio is their number of UFC wins divided by their
total number of UFC fights (including draws).
Table~\ref{table_simple_rules} makes clear the advantage of being young, having
greater reach and being a southpaw.  Note that Conor McGregor has all three of
these traits.

\subsection*{Machine Learning Models}

One limitation of simple rule-based models is that they
cannot be applied to all fights. For instance, if the two fighters
have the same reach, a model that predicts the fighter with the longer
reach to win must exclude the fight.
In this section we go beyond simple rules to machine learning,
which is a branch of Artificial Intelligence.

In the first stage of supervised machine learning, a model
is trained by feeding it information about the two fighters
and which of the two won the bout. After the training phase,
the model is said to be smart and we show it information about
two fighters and ask it to tell us who won.

For each fighter we have the following features:

\begin{itemize}[noitemsep]
  \item Height, $h$
  \item Reach, $r$
  \item Leg reach, $l$
  \item Stance (orthodox, southpaw or switch), $osw$
  \item Date of birth or age, $a$
  \item College graduate, $e$
  \item Current or former champion/interim champion, $c$
  \item Number of UFC fights, $n_f$
  \item Number of UFC wins, $n_w$
  \item Win ratio of UFC fights, $w_r=n_w/n_f$
  \item Total time in the Octogon, $t_T$
  \item Elapsed time since last UFC fight, $t_e$
  \item Fraction of wins by KO/TKO, $W_{\textrm{TKO}}$
  \item Fraction of wins by submission, $W_{\textrm{SUB}}$
  \item Fraction of wins by decision, $W_{\textrm{DEC}}$
  \item Fraction of losses by KO/TKO, $L_{\textrm{TKO}}$
  \item Fraction of losses by submission, $L_{\textrm{SUB}}$
\end{itemize}

\noindent
We also use the following offensive and defensive statistics:

\begin{itemize}[noitemsep]
  \item Significant strikes landed per minute, $s_l$
  \item Significant striking accuracy, $a_s$
  \item Significant strikes absorbed per minute, $s_a$
  \item Significant strike defense, $a_f$
  \item Average takedowns landed per 15 minutes, $d_t$
  \item Takedown accuracy, $d_a$
  \item Takedown defense, $d_f$
  \item Average submissions attempted per 15 minutes, $s_b$
\end{itemize}

\begin{itemize}[noitemsep]
  \item Significant strikes attepted per minute
  \item Knockdowns per minute
  \item Total strikes landed per minute
  \item Total strikes attempted per minute
  \item Total strikes absorbed per minute
  \item Average takedowns attempted per 15 minutes
  \item Average passes per 15 minutes (e.g., half-mount to side-mount)
  \item Average reversals per 15 minutes (e.g., half-guard to full-guard)
\end{itemize}

\noindent
The derived features are

\begin{itemize}[noitemsep]
  \item Difference in age, reach and so on of the two fighters (e.g., $\Delta r=r_1-r_2$)
  \item Total damage absorbed = cumulative sum of significant strikes absorbed
  \item Striking ratio = significant strikes landed / significant strikes absorbed
  \item Natural logarithm of elapsed time since last UFC fight
  \item Fighter versatility index
\end{itemize}

\noindent
The fighter versatility index was introduced by Estelami as
\begin{equation}
FVI = \frac{\frac{1/3}{W_{\textrm{TKO}}^2+W_{\textrm{SUB}}^2+W_{\textrm{DEC}}^2}-\frac{1}{3}}{2/3}.
\end{equation}

The $FVI$ is zero when a fighter achieves all wins in one manner and
it is one when all three methods of victory are uniformly used.

\noindent
These features depend on both fighters:

\begin{itemize}[noitemsep]
  \item Expected strikes absorbed per minute of fighter 1 = $a_{f,1} (s_{l,2}/a_{s,2})$
  \item TKO susceptibility of fighter 1 = $(L_{\textrm{TKO},1} \times W_{\textrm{TKO},2})^{1/2}$
  \item Submission susceptibility of fighter 1 = $(L_{\textrm{SUB},1} \times W_{\textrm{SUB},2})^{1/2}$
  \item Same three above except for fighter 2
\end{itemize}

It is important to note that these quantites are computed
up to the time of the fight. We do not use any information
obtained during the fight or afterwards to make predictions.
Later, we will violate this rule to see if such an approach
leads to an improvement in model performance.

The most important part of any machine learning model is
to assemble a collection of discriminatory features. While
our list is comprehensive, we will find it to be insufficient.

One additional thought not taken into account is to weight
wins, losses, strikes landed and attempt by who the opponent
is. Such weights would need to be found in self-consistent
manner which suggest a cascading scheme.

Some of the featurs are correlated such as reach
and height. We will address this in the next section.


\subsection*{Prediction}

In cases where heigh and reach were known but leg reach
was missing, we imputed values by creating a linear model
based on reach and height. After removing fights where
dates of birth were not known, we were left with 3300 fights,
all of which took place after January 1, 2005.

Before creating our first model we examine a correlation
matrix of certain features. Table~\ref{corr_mat} shows that
most correlation coefficients are small. The largest ones
are for reach and height. This suggests that training a model
on both features will not be useful and may even be
detrimental. We see that there is a postive
correlation between win ratio and significant strikes landed
per minute while the correlation is negative between win ratio
and significant strikes absorbed per minute.

\begin{center}
\begin{table}[h]
\input{prediction/corr_table.tex}
\caption{Correlation matrix for basic features.}
\label{corr_mat}
\end{table}
\end{center}

During the training phase, we randomly select 70\% of the fights
and use this data to fit the models. This is done using
stratified K-fold cross validation with 10 folds. A grid search
procedure is used to find the optimal hyperparameters for each
model. The optimized model is then used to predict the outcome
of the remaining 30\% of the data. The accuracy is then determined
by comparing the predicted outcomes with the known outcomes.
Because results vary with train-test split, we shuffle the
data and repeat the procedure above ten times. The average
of these ten accuracy scores are then recorded.
While using the
test data is consider bad practice, this approach eliminates
concerns associated with using a single split which may give
favorable or unfavorable results depending on how the data
is partitioned.

Random forests (RF) is a tree-based ensemble method,
logistic regression (LR) is an example of the generalized
linear model with the logit link function, the multilayer perceptron (MLP)
is an artificial neural network and AdaBoost is
an ensemble method based on adaptive boosting.
For each model we optimize the hyperparameters
during K-fold cross-validation. For RF the
number of estimators is 100, bootstrapping is used
and the number of features to split on is taken
as the square root of the total number of features.
For LR we optimize the regularization coefficient
as well as the type of regularization ($L_1$ or $L_2$).
The sizes of the hidden layers were set to (5, 2) for
the MLP and $\alpha$, the $L_2$ regularization parameter, was optimized.
Lastly, for AdaBoost, the depth of the decision trees
(base classifiers), the learning rate and the number
of estimators were optimized.
For non-tree based methods, it is necessary to standardize
each feature. The standardizer was fit to the train data
and then used to transform both the train and test data.

Our first set of models is based on age, height,
reach, leg reach and stance. The accuracy scores
for different models and feature sets are show
in Table~\ref{anthropomorphic_features}.
We see that when age is the only feature, LR is gives
an accuracy of 55.6\%. This is very close to the simple
rule model as expected. The best we can do is just more than
56\%. Given the strong correlation between height, reach
and leg reach, it is not surprising that results do not
improve when these features are used together. We see that
age is a discriminatory feature.

\begin{center}
\begin{table}[h]
\begin{tabular}{r|cccc}
  \toprule
   & Random Forest & Logistic Reg. & Multilayer Perceptron \\
  Features & Accuracy & Accuracy& Accuracy \\
  \hline
  $a$ & 53.4\% & 55.6\% & 55.5\%\\
  $r$ & 52.2\% & 52.1\% & 51.9\%\\
  $a$, $r$ & 52.7\% & 56.8\% & 56.1\%\\
  $\Delta a$, $\Delta r$ & 52.9\% & 56.3\% & 56.2\%\\
  $h$, $r$ & 52.9\%  & 52.1\% & \\
  $\Delta h$, $\Delta r$ & 51.6\% & 51.6\%  &\\
  $h$, $\Delta h$, $r$, $\Delta r$ & 52.0\% & 51.9\%  &\\
  $h$, $r$, $l$ & 56.6\% & 51.9\% &  \\
  $\Delta h$, $\Delta r$, $\Delta l$ & 50.7\% & 52.2\%  &\\
  $\Delta h$, $\Delta r$, $\Delta l$, $\Delta a$ & 52.4\% & 55.1\% & 55.4\% \\
  $h$, $r$, $a$ & 53.4\% & 54.7\% &  \\
  $h$, $r$, $a$, $a^2$ & 53.4\% & 55.1\% &  \\
  $r/h$, $a^2$ & 54.3\% & 55.1\% & 55.2\%  \\
  $h$, $r$, $l$, $a$ & 56.1\% & 54.9\% &  \\
  $h$, $r$, $l$, $a$, $\Delta h$, $\Delta r$, $\Delta l$, $\Delta a$ & 55.8\% & 54.9\%  &\\
  $osw$ & 52.3\% & 52.3\% & 52.3\%\\
  $a$, $osw$ & 51.0\% & 55.0\% & \\
  $r$, $a$, $osw$ & 53.5\% & 55.7\% &  \\
  $r$, $l$, $a$, $osw$ & 56.1\% & 55.4\% & \\
  $h$, $r$, $l$, $a$, $osw$ & 56.7\% & 55.3\% &  \\
  $h$, $r$, $l$, $a$, $osw$, $\Delta h$, $\Delta r$, $\Delta l$, $\Delta a$, $\Delta osw$ & 56.0\% & 54.8\% &  \\
  \bottomrule
\end{tabular}
\caption{Accuracy scores for different machine learning classifiers which were trained
on age, stance and anthropometric measurements only.
$a$ is age, $h$ is height, $r$ is reach, $l$ is leg reach,
$osw$ stands for three indicator variables for orthodox, southpaw,
switch and $\Delta$ is the difference in one of the features between the two
fighters. For instance, in the third row the models were trained on
four features or the height and reach of both fighters (i.e., $h_1$, $r_1$, $h_2$, $r_2$)
whereas in the fourth row only two features were used (i.e., $h_1-h_2$ and $r_1-r_2$).}
\label{anthropomorphic_features}
\end{table}
\end{center}

Table~\ref{other_features} gives results for models with more
advanced choices of the features. We see that LR with win ratio
and age as the features (i.e., $w_{r,1}$, $w_{r,2}$, $a_1$, $a_2$) the
accuracy score is 57.5\%. As more featurs are introduced such as
total number of fights, stance, champion status and education, the
accuracy increases somewhat. We only achieve 60\% when the feature
extraction technique called linear discriminant analysis is applied
to the full feature matrix and then LR is used.

\begin{center}
\begin{table}[h]
\begin{tabular}{r|cccc}
\toprule
         & RF       & Logistic Reg. & MLP      & AdaBoost\\
Features & Accuracy & Accuracy      & Accuracy & Accuracy\\
\hline
$w_r$ & 54.0\% & 55.0\% & 55.9\% & \\
$\Delta w_r$ & 52.1\% & 55.5\% & 55.5\% & \\
$w_r$, $n_f$ & 52.7\% & 55.1\% & 53.7\% & \\
$w_r$, $n_f$, $n_w$ & 53.4\% & 55.8\% &  & \\
$w_r$, $a$ & 55.5\% & 57.5\% & 56.6\% & \\
$\Delta w_r$, $\Delta a$ & 53.3\% & 57.6\% & 57.3\% & \\
$n_f$, $n_w$, $a$, $r$ & 55.1\% & 57.0\% \\
$n_f$, $n_w$, $a$, $r$, $h$, $l$, $c$ & 57.8\% & 58.0\% & &\\
$n_f$, $n_w$, $a$, $r$, $h$, $l$, $osw$ & 57.2\% & 58.8\% & & 58.1\%\\
$n_f$, $n_w$, $a$, $r$, $h$, $l$, $osw$, $c$, $e$ & 58.0\% & 58.8\% & &\\
$n_f$, $n_w$, $a$, $r$, $h$, $l$, $osw$, $c$, $e$, $s_a$ & 58.0\% & 58.5\% & & \\
$n_f$, $n_w$, $a$, $r$, $h$, $l$, $osw$, $c$, $e$, $s_a$, $s_l$ & 57.9\% &59.1\% &58.9\% & \\
$n_f$, $n_w$, $a$, $r$, $h$, $l$, $osw$, $c$, $e$, $a_s$ & 58.6\% & 58.4\% \\
$n_f$, $n_w$, $a$, $r$, $h$, $l$, $osw$, $c$, $e$, $\Delta s_r$ & 57.8\% & 58.2\% & 58.4\% & 58.1\%\\
$n_f$, $n_w$, $a$, $r$, $osw$, $c$ & 56.6\% & 58.9\% & 58.5\% \\
$n_f$, $n_w$, $a$, $r$, $osw$, $c$, $h$, $l$ & 57.9\% & 59.0\% & &\\
$a$, $r$, $s_l$ & 54.9\% & 56.2\% & & \\
$a$, $r$, $d_t$ & 54.3\% & 56.1\% & & \\
$a$, $r$, $s_a$ & 55.7\% & 57.1\% & & \\
$s_l$, $s_a$ & 52.9\% & 56.8\% & 56.5\% & \\
all 121 features & 58.1\% & 59.3\% & 59.1\% & \\
all 121 features + LDA &  & 60.1\% & & \\
\bottomrule
\end{tabular}
\caption{$w_r$ is win ratio, $s_l$ is significant strikes landed per minute,
$s_a$ is significant strikes absorbed per minute, $a_s$ is striking accuracy,
$s_r$ is striking ratio, $d_t$ is average number of takedowns per 15 mintues,
$n_f$ is number of previous UFC fights, $n_w$ is number
of UFC wins, $c$ is the champion indicator variable, $e$ is college graduate.}
\label{other_features}
\end{table}
\end{center}

RF provides a way to gauge the importance or how dicriminatory
each feature is. The feature importances for a given feature
set are shown in Figure~\ref{rf_importances}. We see that age
and win ratio are the dominant features with leg reach and reach
also being relevant. We remind ourselves that because reach and leg
are highly correlated, we can think of them as also the same. Stance
plays only a small role which isn't surprising since most fights
take place between two orthodox fighters.

\begin{figure}[h]
\begin{center}
\includegraphics[width=11cm]{prediction/rf_importances.pdf}
\caption{Feature importances of the random forest model with $w_r$,
$a$, $r$, $h$, $l$ and $osw$.}
\label{rf_importances}
\end{center}
\end{figure}

It is surprising that even with 121 features for each fight
and sophisicated machine learning models, we are only able to
predict 60\% of the fights correctly. The explanation for
the difficulty of prediction can be understood by assessing the
quality of the data. Figure~\ref{lack_of_ufc_fights} shows
a plot of the percentage of UFC fights where at least one of the
fighters had $n$ UFC fights or fewer. We see that 71\% of UFC fights
have taken place between fighters with 3 or fewer previous UFC
fights. Note that while we only considered fights after January 1, 2005,
we computed the number of previous fights using all the data. Because
so many fights take place between fighters with so few UFC fights,
metrics such as significant
strikes landed per minute and number of takedowns per 15 minutes
are not reliable. This explains why we are able to only improve on
our model based on age and reach using an addition 117 features.
Of the 3300 fights, 892 fights one or both
fighters are making their debut. Because we ignore their previous
fights in other promotions we know nothing about them.

\begin{figure}[h]
\begin{center}
\includegraphics[width=11cm]{prediction/lack_of_ufc_fights.pdf}
\caption{Percentage of UFC fights where one or both fighters
have only $n$ previous UFC fights or less. For the
`Since 2005' data set with $n=0$, 1, 2, 3, 4, 5 the values are
28.0\%, 48.8\%, 61.5\%, 70.9\%, 77.7\% and 83.1\%, respectively.}
\label{lack_of_ufc_fights}
\end{center}
\end{figure}

To verify our hypothesis we retrained our models using
bouts where both fighters have had a certain number of
UFC fights (see Table~\ref{other_features_with_min}). We
see that when each fighter has at least four fights, we
can achieve an accuracy of slightly greater than 60\%. 
While the improvement is marginal this result lends support
to our hypothesis.

\begin{center}
\begin{table}[h]
\begin{tabular}{ccr|ccc}
\toprule
Min. Number &        &         & RF        & Logistic Reg. & MLP      \\
of Fights & Fights & Features & Accuracy & Accuracy      & Accuracy\\
\hline
0  & 3300 & all 121 features & 58.1\% & 59.3\% & 59.1\% \\
2  & 1740 & all 121 features & 59.9\% & 58.4\% &  \\
4  & 1006 & all 121 features & 58.3\% & 59.8\% &   \\
4  & 1006 & $w_r$ & 52.5\% & 57.4\% & 57.4\%  \\
4  & 1006 & $w_r$, $a$ & 56.9\% & 58.9\% & 56.7\%  \\
4  & 1006 & $w_r$, $a$, $r$, $h$, $l$, $osw$ & 57.9\% & 58.6\% & 56.0\%\\
4  & 1006 & $w_r$, $a$, $s_l$, $s_a$ & 59.0\% & 60.0\% & 59.9\%  \\
4  & 1006 & $w_r$, $a$, $d_t$, $d_f$ & 57.4\% & 58.0\% & 56.1\%  \\
4  & 1006 & $w_r$, $a$, $s_r$ & 59.1\% & 60.8\% & 61.0\%  \\
4  & 1006 & $n_w$, $w_r$, $a$, $r$, $h$, $l$, $osw$, $c$, $e$, $s_r$, $s_l$, $s_a$, $d_t$ & 60.0\% & 59.8\% & 56.5\%  \\
6  &  586 & all 121 features & 59.9\% & 60.0\% & 54.5\% \\
8  &  343 & all 121 features & 58.5\% & 61.1\% & 54.0\% \\
\bottomrule
\end{tabular}
\caption{Model performance when only fights with experienced
fighters are considered. $d_f$ is average takedown defense}
\label{other_features_with_min}
\end{table}
\end{center}

To confirm this position, we have computed the career-average of
each fighters statistics and  used those as the features instead
of the moving averages, which we just argued are not very useful
since most fighters have so few fights. When these ``from the future''
values are used, we obtain the results
in Table~\ref{scores_using_career_stats}. Indeed, the career-averaged
values improve the performance of the model markedly. However, they
are off limits to us in reality. This tells us that fighters do
get better over time. To see this we plotted the squared deviation
versus normalized fight number in Figure~\ref{normalized_sl_dist}
for the 250 UFC fighters with the most fights. In the right panel
the average curve is shown. Indeed the curve relaxes to zero over
a non-trivial number of fights.

\begin{center}
\begin{table}[h]
\begin{tabular}{r|ccc}
\toprule
         & RF       & Logistic Reg. & MLP     \\
Features & Accuracy & Accuracy      & Accuracy\\
\hline
all 145 features & 66.0\% & 66.3\% & 66.0\% \\
8 career stats & 65.6\% & 66.3\% & 66.2\% \\
$\hat{s}_{l,\textrm{FM}}$, $\hat{s}_{a,\textrm{FM}}$ & 62.5\% & 64.4\% & 66.0\% \\
$\hat{s}_{l,\textrm{UFC}}$, $\hat{s}_{a,\textrm{UFC}}$ & 62.6\% & 64.6\% & 64.3\% \\
$s_l$, $s_a$ & 52.9\% & 56.8\% & 56.5\% \\
$\hat{w}_r$ & 75.7\% & 79.6\% & 79.4\% \\
$w_r$ & 54.0\% & 55.0\% & 55.9\% \\
\bottomrule
\end{tabular}
\caption{Accuracy scores when future data is used
to compute the eight career statistics, which are then
used as features.
The eight career-averaged statistics are
significant strikes landed per minute,
significant striking accuracy,
significant strikes absorbed per minute,
significant strike defense  (the percentage of opponents strikes that did not land),
average takedowns landed per 15 minutes,
takedown accuracy,
takedown defense (the percentage of opponents TD attempts that did not land),
average submissions attempted per 15 minutes. $\hat{s}_{l,\textrm{FM}}$ is
the career-average value of significant strikes landed per minute
taken from FightMetric, which
includes fights outside the UFC, whereas $\hat{s}_{l,\textrm{UFC}}$ is
the career-average value computed using only UFC fights.
$s_l$ and $s_a$ are computed using only UFC data before the fight. $\hat{w}_r$
is the career-averaged win ratio.}
\label{scores_using_career_stats}
\end{table}
\end{center}



\begin{figure}[h]
\begin{center}
\includegraphics[width=15cm]{prediction/sl_dist.pdf}
\caption{(left) Square of the difference between the moving average
and overall average of significant strikes landed per minute as
a function of normalized fight index for the top 250 fighters
with the most UFC fights. $n$ is the $n$-th UFC fight of
a given fighter and $n_f$ is their total number of UFC fights.
Note that $\bar{s}_l=\hat{s}_l$ at $n/n_f=1$ for all fighters.
(right) Same as left panel except averaged over all 250 data sets.}
\label{normalized_sl_dist}
\end{center}
\end{figure}

